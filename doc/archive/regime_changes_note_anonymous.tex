\documentclass[12pt,final,fleqn]{article}

% basic packages
\usepackage[margin=1in] { geometry }
\usepackage{amssymb,amsmath, bm}
\usepackage{verbatim}
\usepackage[latin1]{inputenc}
%\usepackage[OT1]{fontenc}
\usepackage{setspace}
\usepackage{enumitem}
\usepackage{url}
\usepackage[font={bf}]{caption}
%\usepackage{pgfplots}
%\usepackage[font={bf}]{caption}
\usepackage{setspace}
\usepackage{latexsym}
%\usepackage{euscript}
\usepackage{graphicx}
\usepackage{marvosym}
%\usepackage[varg]{txfonts}  Older version of ``g'' in math.

% bibliography packages
\usepackage[natbibapa]{apacite}
\bibliographystyle{apsr}
\bibpunct{(}{)}{;}{a}{}{,}
\renewcommand{\bibname}{References}

% Footnote double spacing and font size for journal submission
%\usepackage{footmisc}
%\setlength{\footnotesep}{\baselineskip}
%\renewcommand{\footnotelayout}{\normalsize\doublespacing}

% hyperref options
\usepackage{color}
\usepackage{hyperref}
\usepackage{xcolor}
\hypersetup{
    colorlinks,
    linkcolor={blue!50!black},
    citecolor={blue!50!black},
    urlcolor={blue!80!black}
}
\newcommand*{\Appendixautorefname}{Appendix}
\renewcommand*{\sectionautorefname}{Section}
\renewcommand*{\subsectionautorefname}{Section}
\newcommand{\aref}[1]{\hyperref[#1]{Appendix~\ref{#1}}}

\setcounter{secnumdepth}{0}

% packages for tables
\usepackage{longtable}
\usepackage{booktabs, threeparttable}
\usepackage{threeparttablex}
%\usepackage{tabularx}
% dcolumn package
\usepackage{dcolumn}
\newcolumntype{.}{D{.}{.}{-1}}
\newcolumntype{d}[1]{D{.}{.}{#1}}
\captionsetup{belowskip=10pt,aboveskip=-5pt}
\usepackage{multirow}
% rotating package
\usepackage[figuresright]{rotating}
\usepackage{pdflscape}
\usepackage{subcaption}

% packages for figures
\usepackage{grffile}
\usepackage{afterpage}
\usepackage{float}
\usepackage[section]{placeins}
\usepackage[export]{adjustbox}

% theorem package
\usepackage{theorem}
\theoremstyle{plain}
\theoremheaderfont{\scshape}
\newtheorem{theorem}{Theorem}
\newtheorem{algorithm}{Algorithm}
\newtheorem{assumption}{Assumption}
\newtheorem{lemma}{Lemma}
\newtheorem{proposition}{Proposition}
\newtheorem{remark}{Remark}
\newcommand{\qed}{\hfill \ensuremath{\Box}}
\newcommand\indep{\protect\mathpalette{\protect\independenT}{\perp}}
\DeclareMathOperator{\sgn}{sgn}
\DeclareMathOperator{\tr}{tr}
\DeclareMathOperator{\argmin}{arg\min}
\DeclareMathOperator{\argmax}{arg\max}
\def\independenT#1#2{\mathrel{\rlap{$#1#2$}\mkern2mu{#1#2}}}
\providecommand{\norm}[1]{\lVert#1\rVert}
\renewcommand\r{\right}
\renewcommand\l{\left}
\newcommand\E{\mathbb{E}}
\newcommand\dist{\buildrel\rm d\over\sim}
\newcommand\iid{\stackrel{\rm i.i.d.}{\sim}}
\newcommand\ind{\stackrel{\rm indep.}{\sim}}
\newcommand\cov{{\rm Cov}}
\newcommand\var{{\rm Var}}
\newcommand\SD{{\rm SD}}
\newcommand\bone{\mathbf{1}}
\newcommand\bzero{\mathbf{0}}

% dotted lines in tables
%\usepackage{arydshln}

\usepackage{pdflscape}

% spacing between sections and subsections
\usepackage[compact]{titlesec}

% times new roman
%\usepackage{times}

% appendix settings
\usepackage[toc,page,header]{appendix}
\renewcommand{\appendixpagename}{\centering Appendices}
\usepackage{chngcntr}
\usepackage{etoolbox}
\usepackage{lipsum}

% Endnotes for CP
%\usepackage{endnotes}
%\let\footnote=\endnote


% file paths and definitions
\makeatletter
\newcommand*\ExpandableInput[1]{\@@input#1 }
\makeatother

\setlength{\mathindent}{1cm}
\allowdisplaybreaks[4]
\doublespacing
%\special{pdf: pagesize width 8.5truein height 11.0truein}

\titleformat{\subsection}
  {\itshape\large}{\thesubsection}{1em}{}

\setcounter{tocdepth}{1}
\begin{document}
%\author{Devin Incerti\thanks{Principal Data Scientist, Genentech. devin.incerti@gmail.com} and Trevor Incerti\thanks{Corresponding author. PhD Candidate in the Department of Political Science, Yale University. trevor.incerti@yale.edu.}}
\title{\textbf{Does political instability always harm investment? Evidence from financial data}}
\date{\today}
\maketitle
\thispagestyle{empty}

\singlespacing
\begin{abstract}
\noindent
Political instability is commonly thought to discourage investment and reduce economic growth. We challenge this consensus by showing that instability does not systematically depress access to financial capital. Using an event study approach, we examine daily returns of national financial indices in every country that experienced an irregular regime change subject to data availability. Returns following resignations are large and positive (+4\%), while those following assassinations are negative and smaller in magnitude (-2\%). The impact of coups tends to be negative (-2\%), but we show that a pro-capitalist coup results in large positive returns (+10\%). We also find evidence that authoritarian or anti-capitalist regime changes are more likely to lead to capital flight than democratic or pro-business changes.  The immediate impact of political instability on capital access is therefore dependent on the type of regime change and its expected impact on future growth. 

\end{abstract}
%\doublespacing

\clearpage
\pagenumbering{arabic}

%\section{Introduction} \label{sec:Introduction}
\newpage

Economies are now global, financialized, and integrated, with domestic economies intrinsically linked to international investment. However, political instability appears to jeopardize access to capital---instability is negatively correlated with investment, financial development, and GDP growth \citep{aisen2013does, alesina1996income, alesina1996political, fosu1992political, jong2009measurement, roe2011political, baker2016measuring}, as well as associated with increases in stock market variance \citep{leblang2005government, jensen2005market, liu2015economic}.

% Investors rank political risk as a top consideration when investing in emerging markets \citep{wipr2011, wipr2012, wipr2013}.

%Unstable governments have been argued to be less likely to invest in the legal system and the protection of property rights \citep{svensson1998investment}, and to be more likely to increases taxes \citep{devereux1998political}. 

%---as measured by regime change frequency, political violence, or indices of uncertainty derived from news reports---

However, how different \textit{types} of political instability affect domestic capital access remains unclear. In contrast to the aforementioned cross country studies, we therefore separately examine the effect of different types of instability on domestic firms' access to financial capital. Specifically, we examine whether changes in financial flows differ for coups, resignations, assassinations, and protests, as well as for authoritarian vs. democratic and pro vs. anti business shifts. 

To test whether there are meaningful differences in capital flows following different types of political instability, we examine changes in stock market returns surrounding politically unstable domestic events. Specifically, we conduct event studies of daily financial data, which estimate a local average treatment effect of an unexpected event on stock prices \textit{at the time of the event}. This interrupted time-series approach mitigates the endogeneity problems in previous cross country regressions---confounding events would need to occur on the same day as instability, and do so for a large portion of all of our independently tested events in order to influence our estimates.

%Stock market returns are used as an indicator of whether investors view different types of potentially destabilizing political events as ``good'' for future returns. Beyond the large changes in real wealth created by extreme stock market volatility, empirical research has tied future production growth rates to variation in stock returns \citep{schwert1990stock, fama1990stock}.

%An ``abnormal return'' therefore reflects the difference between the observed return and the predicted return. We use daily data as it allows for more precise measurement of abnormal returns. 

We analyze the full sample of politically unstable events for which national-level daily financial data is available.\footnote{13 coups, 8 assassinations, 15 forced resignations, and 11 public protests.}  Like previous research, we find that all types of political instability in our sample cause large increases in financial volatility. However, we find that market returns are large and positive (+4\%) following resignations and negative and smaller in magnitude following assassinations and coups (-2\%). We also find that the failed pro-business 2002 Venezuelan coup caused large positive market returns (+10\%), followed by immediate capital flight (-8\%) after the reinstatement of a left-wing populist.

Our primary contributions are empirical and methodological. We provide the first estimates of the effects of different types of political instability on domestic capital access. We show that instability does not systematically depress investment, and find evidence that authoritarian regime changes are more likely to lead to negative returns, but that leaders who are clearly pro-business can be rewarded by financial markets even if they use extra-judicial methods to take power. The capital flows we document are not insubstantial---they are in some cases larger shocks than the 2008 stock market crash on their respective domestic economies. Methodologically, we (1) employ a method less susceptible to endogeneity concerns than previous studies, and (2) integrate synthetic control and event study methods to allow for control portfolios when a control candidate is not present.

%Our primary contributions are empirical and methodological. Empirically, we provide the first estimates of the effects of different types of political instability on domestic capital access, and show that instability does not systematically depress investment. The capital flows we document are not insubstantial---they are in many cases larger shocks than the 2008 stock market crash on their respective domestic economies. Methodologically, we (1) employ a method less susceptible to endogeneity concerns than previous studies, and (2) integrate synthetic control and event study methods to allow for control portfolios when a control candidate is not present. Theoretically, we find evidence that authoritarian regime changes are more likely to lead to negative returns, but that leaders who are clearly pro-business can be rewarded by financial markets, even if they use extra-judicial methods to take power.

%Investors recognize when instability may be in their favor.

%Our study therefore not only uses an event-study approach to enhance the reliability of our estimates, but simultaneously examines a relatively large sample that enhances external validity and allows us to show that different types of political events have disparate effects. 

\section{Reexamining capital response to political instability}

Stable capital flows are highly important to financial stability in emerging markets, which are particularly exposed to shifts in the availability of foreign capital \citep{koepke2019drives, obstfeld2012financial, cohen2017global}. High country risk reduces capital flows, which in turn has been shown to reduce domestic output growth \citep{koepke2019drives}. 

Conventional theory and cross-country empirical evidence suggests that political instability increases country risk and therefore depresses financial returns and economic growth \citep{irshad2017relationship, le2006political, lensink2000capital, boutchkova2012precarious, lehkonen2015democracy}, and increases volatility \citep{irshad2017relationship, bialkowski2008stock, leblang2005government, jensen2005market, liu2015economic}. But if capital flows reflect expectations of future economic growth and returns, such hypotheses may be too simplistic. For example, capital access may increase in response to a coup if the current regime is anti-business or anti-global.

%Previous research therefore suggests that coups should on average not lead to positive economic development or market returns, but that ``good coups'' could occasionally create positive returns. As our event study approach allows us to analyze individual coups separately from others, we can test this prediction. 

Not all irregular regime changes are equivalent. Some---e.g., the resignation of an ineffective leader---may foreshadow better policy. Assassinations can occur seemingly at random, and a successor may be unclear. Coups can be democratic or autocratic. These differences are often overlooked in past studies, which have proxied for instability using the number of coups \citep{londregan1990poverty, alesina1996political}, assassinations or revolutions \citep{barro1991economic}, or combined events into single indices \citep{alesina1996income, venieris1986income, gupta1990economics, jong2009measurement}. 

% \footnote{These studies vary by metrics included in the indices, method of aggregation, and outcome variable of interest.}

Debate also exists on coups that lead to democratization or economic liberalization and their effect on economic growth.  Most research suggests these are not the norm \citep{derpanopoulos2015coups, powell2011global, thyne2016coup, varol2011democratic}. Others argue the negative effects of uncertainty dominate any positive effects \citep{alesina1996political}, or that the impact of coups is neutral as some enhance growth while others depress it \citep{londregan1990poverty}. But while these ``good coups'' may be the minority, they have become more frequent \citep{marinov2014coups} and could have positive economic effects \citep{meyersson2016political, girardi2018institution}. Regular regime changes such as elections have been found to boost markets following opposition victories in weak democracies \citep{pantzalis2000political}, and coups may exhibit the same pattern. 


%Different types of regime changes may therefore have disparate effects on markets and/or economic growth. 

Previous research therefore suggests coups may on average cause capital flight, but lead to positive flows if the coup's instigators are more democratic or pro-market than the regime they replace.  Assassinations should have a neutral or negative impact as they increase uncertainty, but institutional responses to assassinations vary by country. By contrast, resignations may be viewed positively on average, as they often signal the departure of an ineffective leader. 

%Our empirical findings validate these predictions. We find that the type of political event and its expected impact on economic policy determines the direction of abnormal returns. Events expected to lead to more stable governance, economic liberalization, or democratization (such as willful resignations and coups that overthrow protectionist or leftist autocrats) are associated with positive returns, while those that consolidate authoritarian rule (e.g. military coups), exacerbate poor economic policies, or merely increase policy uncertainty (such as assassinations) have the opposite effect.

\section{Data}

%---the longest available daily time series of stock prices

Financial data is from the Global Financial Data database. We collect national stock index data for every country in which there was a coup, assassination, or resignation and daily financial data is available.\footnote{The list of failed assassinations are from \citet{jones2009hit}. Coup attempts are those in category 2 in the CSP Coup d'etat dataset.} These stock indices are listed in \autoref{tab:stock-list}.

Political data are drawn from the Center for Systemic Peace's (CSP) Polity IV Coup d'etat dataset and Coup d'etat Events handbook. We form a list of ``irregular'' regime changes from successful coups,\footnote{For example, \citet[p. 617]{needler1966political} states that ``the categories of coups that were aborted, suppressed, or abandoned melt into each other and into a host of other non-coup phenomena so as to defy accounting,'' the CSP is ``confident that [its] list of successful coups is comprehensive'' but does not extend this confidence to attempted or failed coups, and \citet{powell2011global} state that it is ``difficult to identify more ambiguous forms of coup activity, such as coup failures, plots, and rumors.''} assassinations of the executive, and resignations of the executive\footnote{Resignations are those in which the ruling executive was coerced to resign due to poor performance, public discontent and popular demonstrations.Note that the Polity IV definition of ``poor performance'' is not synonymous with poor \textit{economic} performance, and in practice the reasons cited for resignation across events are: loss in conflict/war, anti-authoritarian protest, corruption scandals, Supreme Court ruling against unconstitutional actions, contested elections, and abuse of power.} as daily financial data is available for countries in these categories. We supplement the CSP data with Archigos Version 4.1 leadership data, which allows us to identify some additional cases of coups, assassinations, and protests. A list of the political events in our dataset and whether they were democratic or autocratic shifts is shown in \autoref{tab:regime-changes}. 

%in which a ``leader lost power through irregular means.'' Irregular transfers of power are those in which leaders do not leave office ``in a manner prescribed by either explicit rules or established conventions.'' Nearly all removals by irregular means result from the threat or use of force (e.g. coups, revolts and assassinations). 

There is considerable debate about classification of regime changes. We recognize that some readers may feel certain events are missing. We rely on common third-party classifications to minimize the possibility our results are driven by our own classifications. The exception is that we separately analyze the 2002 failed coup  in Venezuela, as it provides a test of the impact of the seemingly successful removal of a left-wing populist with a pro-business regime, and the ensuing reinstatement of a left-wing populist. 

%in which President Hugo Chavez was removed from office for two days

%\footnote{Defined by CSP as the ``occurrence of subversion of the constitutional order by a ruling (usually elected) executive and the imposition of an autocratic regime."}

%loss in conflict/war (Argentina), anti-authoritarian protest (Philippines, Bangladesh, Nepal, Tunisia), corruption scandal (Philippines, Lithuania), constitutional ruling (Pakistan), contested election (Ukraine), abuse of power (Ecuador) Thailand?

%Coups tend to have the largest impact on the level of democratization as a number of countries have subsequently transitioned from democracies to anocracies or autocracies. On the other hand, neither assassinations nor resignations tend to have much impact on the level of democratization. 

%\autoref{fig:AV-DR} plots the absolute value of daily stock returns averaged across all events. The returns on the event day (day 0) are significantly larger than on any other day. The magnitude of returns begins increasing just before the event day and remains high for a short period after. This suggests that financial volatility increases during the days surrounding regime changes, but says nothing about mean returns. \nameref{sec: Impact of Political Instability on Stock Returns} will first formally estimate the volatility surrounding all irregular regime changes, then analyze coups, assassinations and resignations separately and determine their impact on mean returns.

\section{Estimation} \label{sec: estimation}

\subsection{Volatility} \label{subsec: Volatility}

To estimate the effect of irregular regime changes on financial volatility, we use a generalized autoregressive conditional heteroskedasticity (GARCH) model estimated using 1000 pre-event days, the event day and 1000 post-event days. As in \citet{jensen2005market} and \citet{leblang2005government}, we use the GARCH (1,1) specification. In particular, for national stock index $i$,
\begin{align*}
R_{it}=\mu_i + \epsilon_{it},\hspace{1cm} \epsilon_{it}\sim \mathcal{N}\left(0,\sigma_{it}^2\right),
\end{align*}
where $\mu_i$ is a constant and,
\begin{align*}
\sigma_{it}^2&=\gamma_{i}+\alpha_{i}\epsilon_{i,t-1}^2+\beta_{i}\sigma_{i,t-1}^2.
\end{align*}

The key parameter of interest is the conditional variance, $\sigma_{it}^2$. The one-period-ahead volatility forecasts, $\sigma_{it}$, are larger when $\epsilon_{i,t-1}^2$ and $\sigma_{i,t-1}^2$ are larger. In other words, the model predicts that large shocks will be followed by other large shocks.

\subsection{Abnormal returns} \label{subsec:abnormal-returns}

%\footnote{A growing body of work uses event studies to assess political phenomenon.  For example, political events have been used to estimate the effect of political connections on firm value \citep[e.g.,][]{fisman2001estimating,faccio2006politically,goldman2009politically}.} 

%  the magnitude and direction of

To estimate the effect of irregular regime changes on financial flows, we follow standard event study methodology \citep{mackinlay1997event, campbell1997econometrics}. Normal performance is measured with a constant mean return model,
\begin{align} \label{eqn:market-model}
R_{it}=\mu_{i}+\epsilon_{it},
\end{align}
where $R_{it}$ is the logged return of national stock index $i$ on trading day $t$ and $\epsilon_{it}$ is the error term. We calculate abnormal returns (ARs) in an ``event window'' around the date of each event, $AR_{i\tau}=R_{i\tau}-\widehat{\mu}_i$, where $\tau$ is a date in the event window, and $\widehat{\mu}_i$ is estimated in an ``estimation window'' preceding the event window with \autoref{eqn:market-model}. We use a 41 trading day event window (20 pre-event days, event day, and 20 post-event days) and 250 trading day estimation window. The ARs are then used to calculate cumulative abnormal returns (CARs) between event day $\tau_1$ and event day $\tau_2$: $CAR(\tau_1,\tau_2)=\sum_{\tau=\tau_1}^{\tau_2}AR_{i\tau}$. Standard errors and p-values are calculated using asymptotic t-statistics as in \citet{mackinlay1997event}.\footnote{This is appropriate because the length of the estimation window is sufficiently long (250 trading days).} The event date is the first trading day a market could react to news of the event.

%\footnote{In other words, for cumulative abnormal returns prior to the event date, we aggregate backwards starting at the day of the event. For example, $CAR(-1,-2$) is the sum of the abnormal returns on event date $-1$ and event day $-2$.} 

%If an event occurred on a weekend, the event date is the following Monday. 

We do not use a market model (where a market index is included as a control) as our unit of analysis \textit{is} the country-wide market index (i.e., not a firm). To address concerns regarding use of a constant mean return model, we combine synthetic control methods \citep{abadie2010synthetic} with event study estimation. We create a ``synthetic'' control portfolio for each event, where each country is given a weight representing its influence in the portfolio. The weight is chosen so that the daily returns and the variance of the daily returns of the control portfolio and the event country are most similar in the estimation window. The possible countries in the control portfolio are all countries listed in \autoref{tab:stock-list}. A more formal exposition of the methodology can be found in \nameref{subsec: synth}. 

This interrupted time series approach implies that the ``control group'' being compared against is not, for example, regular regime\footnote{Past studies have examined market reactions to regular regime changes. This work finds that volatility increases around elections---particularly close elections and those that lead to a change in government political orientation \citep{bialkowski2008stock} ---and that positive returns tend to precede elections \citep{pantzalis2000political}.} changes or failed regime changes, but the expected returns in the same country on the same day in a but-for world in which no regime change occurred. We report abnormal returns separately for coups, assassinations and resignations. However, each event CAR is by itself also a valid causal estimate of the effect of a specific regime change. 

% to maximize the number of observations and

%and the results are presented in \nameref{subsec: Robustness}.

\section{Impact of Political Instability on Stock Returns} \label{sec: Impact of Political Instability on Stock Returns}

\subsection{Volatility} \label{subsec: volatility}

We first confirm that our sample of events exhibits the increases in volatility suggested by previous literature. \autoref{fig:volatility} shows the mean volatility ($\overline{\sigma_t}$) estimates across all irregular regime changes for 250 trading days prior to and 250 days after each event. Volatility stays between a narrow range at nearly all dates except those surrounding the regime change. There is an enormous volatility jump on the day of the regime change, with levels then decreasing to normal within a month of the event.

%Volatility appears to increase slowly just before the regime change, albeit not to a degree out of line with previous fluctuations. This may suggest that investors sometimes have information about the events before they occur.

\subsection{Coups} \label{subsec: Coups}

\autoref{fig:mean-car-by-regime-change} shows the mean CARs by each type of regime change we analyze for the event day, as well 20 days before and after the event. Individual results for all coups in our sample and mean results for all coups can be found in \autoref{tab:AR-coups}. 

\begin{figure}[!htb]
\centering
\includegraphics[scale = 0.75]{../figs/mean-car-by-regime-change-type.pdf}
\caption{Mean cumulative abnormal returns by type of regime change}\label{fig:mean-car-by-regime-change}
\end{figure}

The average coup has a -2.1\% event day AR. ARs for the 1970 coup in Argentina,  1991 coup in Thailand, 1992 coup in Peru, and 1999 coup in Pakistan are all negative and significantly different than zero. Moreover, all of these cases except Thailand have negative pre-event CARs that are statistically indistinguishable from zero, suggesting they were not foreseen. In all of these cases, the coup either overthrew a democratically elected government or changed governance from one military ruler to another. This initial capital flight followed by additional post-event negativity is consistent with the expected market reaction from a successful authoritarian coup followed by post-event consolidation of power. 

%  post-event CARs and

The only events with positive ARs are the 1971 coup in Argentina and the 2002 coup in Nepal. These results provide evidence that coups do not necessarily lead to negative abnormal returns. While the 1971 Argentinian coup did result in another military leader, it did so while calling for free and democratic elections and replaced a government that had adopted extreme protectionist economic policies. In fact, by 1973 Argentina had transitioned to a democracy.\footnote{Based on Center for System Peace Polity IV polity score of 6. Values of 6-10 are defined as democracies.} The 2002 coup in Nepal resulted in a monarchical restoration, but occurred after the prime minister postponed general elections, itself a democratically subversive action.  

\subsection{Assassinations} \label{subsec: Assassinations}

Like the majority of coups, we find that assassinations lead to capital flight (see \autoref{fig:mean-car-by-regime-change} and \autoref{tab:AR-coups}). The mean event day abnormal return is negative and statistically different than zero. 

%However, the result is driven by five events: the shooting of U.S. President William McKinley on September 6, 1901; the assassination of U.S. President John F. Kennedy on November 22, 1963; the assassination of Indian Prime Minister Indira Gandhi on October 31, 1984; the suicide bombing that killed Sri Lankan president Ranasinghe Premadasa on May 1, 1993; and the assassination of Israeli Prime Minister Yitzhak Rabin on the evening of November 4, 1995.

These results are consistent with our hypothesis that assassinations should have a negative effect as they occur seemingly at random\footnote{There is no evidence of post or pre-event CARs in almost any of the assassinations, consistent with the expectation that assassinations are typically not predictable.} and increase uncertainty. While the mean effect of assassinations is negative, it is smaller in magnitude than for coups. Unlike a coup, an assassination may not necessarily be expected to cause immediate change in economic policy, particularly in the presence of an institutionalized line of succession. As such, we would expect CARs to be negative due to increased instability and uncertainty, but smaller in magnitude to a coup or resignation due to greater expectations of policy inertia.

%As with coups, the number of days that it took the stock market to rebound to pre-event levels is fairly low.\footnote{One exception is the assassination of William Mckinley in which the stock market didn't fully recover for 963 days. However, this was likely caused by the Panic of 1901, which began when the stock market crashed on May 17th, 1901, and not by McKinley's death.}

% (although the assassination may have exacerbated the panic)

\subsection{Resignations} \label{subsec: Resignations}

Unlike coups and assassinations, abnormal returns following resignations are large and positive (see \autoref{fig:mean-car-by-regime-change} and \autoref{tab:AR-resignations}). The mean event day abnormal return is over 4\%, and the positive returns grow larger over time (mean 20-day CAR $\approx$ 12\%). Event day ARs are only negative and significant at even the ten percent level in two of the fifteen resignations.

These results are consistent with our hypothesis that resignations may lead to increased capital access as they typically occur due to poor performance and/or loss of authority. Among our sample of events, leaders were ousted following loss in conflict/war, anti-authoritarian protest, corruption scandals, Supreme Court ruling against unconstitutional actions, contested elections, and abuse of power. 

%For example, the resignation of Ferdinand Marcos as President of the Philippines in February 1986 was associated with an approximately 13\% positive event day AR. Prior to his resignation, the Philippine regime was the least preferred site for foreign investment amongst East Asian capitalist economies due to rampant corruption, high unemployment, failed import substitution industrialization policy, and oligarchic control of the economy \citep{overholt1986rise, traywick2014}. 

%In fact, the Philippines was the least preferred site for foreign investment amongst East Asian capitalist economies \citep{overholt1986rise}.

%For example, consider Ferdinand Marcos' resignation from office as President of the Philippines in February 1986. Prior to his resignation, the Philippine regime was known for rampant corruption, crony capitalism, extreme inequality, high unemployment, failed import substitution industrialization policy, and oligarchic control of the economy \citep{overholt1986rise, traywick2014}. In fact, the Philippines was the least preferred site for foreign investment amongst East Asian capitalist economies and possessed one of the worst capital investment to economic output ratios in Asia \citep{overholt1986rise}. Marcos held a snap presidential election on February 7, 1986, in which he declared victory despite overwhelming evidence of electoral fraud. Public protests ensued, and two weeks later the military withdrew its support of the Marcos regime \citep{lee2009armed}. Marcos was replaced by his electoral opponent, Corazon Aquino, who had run on a platform of economic liberalization and elimination of crony capitalism \citep{villegas1987philippines}. This event was associated with an approximately 13\% positive event day AR.

%By contrast, the largest negative event in our sample (-3\%) is the 1993 resignation of President Ghulam Ishaq Khan and Prime Minister Nawaz Sharif in Pakistan. The resignations occurred after months of political infighting when the army demanded the President and Prime Minister resign and call for new elections. An interim prime minister was installed, but uncertainty about Pakistan's political and economic future remained high prior to the next round of elections.


The resignations analyzed encompass leaders who left office because of poor performance, public discontent and popular protests. Protests preceding these resignations may therefore also affect capital access,\footnote{Indeed, corporate investors in the 2013 MIGA \textit{World Investment and Political Risk} ranked civil disturbances as the fourth most concerning type of political risk.} so we analyze all protests that preceded the resignations in \nameref{subsec: Public Protests}. Including directionality, public protests appear to have no effect on returns as some increase stock prices while others decrease them (see \autoref{tab:protest-stocks}), but in absolute terms, returns are approximately 1.5\% higher during public protests.

%\footnote{The set of resignations includes all those listed in either the Coup d'etat Events Handbook or the Archigos Version 4.1 data set with available financial data. In practice, this is the 2011 Egyptian Revolution and the list of resignations in \autoref{tab:AR-resignations}.} 

%\footnote{These results hold when controlling for emerging market index fluctuations.}


\section{Mechanisms} \label{subsec: mechanisms} 

While investors may generally dislike political instability, the immediate effect of regime changes on financial flows may not always be unpredictable. For example, investors may generally value democracy if it is perceived to provide stronger property rights and lower susceptibility to capital appropriation \citep{przeworski1982structure, north1989constitutions, svensson1998investment}. Investors may also have priors about a new leader's economic ideology. Two mechanisms that may therefore drive the direction of capital flows are: (1) whether the regime change is associated with an authoritarian or democratic shift, and (2) whether a new leader is clearly more pro or anti business than their predecessor. 

We therefore aggregate the events in our sample by whether they resulted in an authoritarian or democratic shift.\footnote{As defined by the Polity project.} We find suggestive evidence that regime changes associated with authoritarian shifts are on average perceived negatively by investors (see \autoref{tab:AR-auth}), while democratic shifts are perceived positively (see \autoref{tab:AR-dem}). Of ten authoritarian shifts, only two result in positive CARs, and neither are significantly different from zero. Three of five cases involving democratic shifts result in positive CARs, and the positive CARs persist in the follows weeks. Of the two negative CARs, one is not significantly different from zero, and the other is associated with a forced market closure that lasted 17 days. While this evidence is suggestive, it comports with previous research showing that political risk to markets declines with higher levels of democracy \citep{lehkonen2015democracy}, and that markets tend to react favorably to opposition victories in weak democracies \citep{pantzalis2000political}.

%However, the majority of the positive returns from democratic regime changes come from the 12\% positive CAR associated with the resignation of Ferdinand Marcos in the Philippines in 1986. 

%We refer to this evidence as suggestive despite statistical significance due to the small sample of cases which fit these criteria, particularly with regard to democratic shifts.

A similar analysis is not possible for pro and anti business shifts, as examples of clear shifts in leader economic ideology do not exist in our sample.\footnote{Based on matching our cases with codings from the The Ideology of Heads of Government (HOG) database, as well as surveys of news reports on the day of each event. In all cases, no clear economic ideological shift can be identified.} We therefore look outside our sample and examine a (failed) pro-business coup followed by the reinstatement of a socialist leader: the 2002 failed coup against Hugo Chavez in Venezuela. This provides a natural test of the effects of both pro and anti business regime changes separately from simple uncertainty because investors reacted to a regime change twice: first, when Chavez was ousted, and second, when he was reinstated. 


\begin{figure}[!ht]
\begin{centering}
\includegraphics[max size={0.8\textwidth}{.8\textheight}]{../figs/venezuela_coup_attempt_2002.pdf}
\caption{Abnormal returns surrounding the 2002 Venezuelan coup attempt}
\label{fig:AR-Ven}
\end{centering}
\end{figure}

%The ultimately failed Venezuelan coup against Hugo Chavez replaced a left-wing populist government with a new pro-business president, and therefore provides a natural test of the effects of both pro-business and anti-business regime changes separately from simple uncertainty because investors reacted to an expected regime change twice: first, when Chavez was ousted, and second, when he was reinstated. On the evening of April 11, 2002, coup plotters removed Chavez from office and later detained him. Pedro Carmona, a Venezuelan economist and business leader, was named the transitional President of Venezuela. Two days later, on April 13, 2002, a popular uprising led to Chavez's reinstatement as president. This provides an estimate of the market's valuation of a transition from the Chavez regime to the Carmona regime and its valuation of a transition from the Carmona regime back to the Chavez regime. By extension, it provides an estimate of the impact of a shift from a left-wing populist government to a pro business regime in an emerging market. 

\autoref{fig:AR-Ven} shows CARs and 95\% confidence intervals for the 20 days surrounding the event. The abnormal return on the first trading day investors could react to the coup was +10\%. When Chavez was reinstated, the abnormal return was -8\%. The 0\% 10-day CAR preceding the coup implies that investors were completely unaware of the coup plot, increasing our confidence that the abnormal returns capture the true effect of the regime changes. This failed coup demonstrates a large influx of capital in response to the attempted overthrow of a socialist leader, and equally large capital flight following his reinstatement.  The large magnitudes and precision of these effects suggest that investors value transition to a pro-business government regime, regardless of how the regime change is achieved. 

%We also note that Chavez's own failed attempt to overthrow the democratically elected government of Carlos Andrés Pérez---at the time pursuing deficit reduction efforts in order to obtain IMF loans---in 1992 was similarly viewed negatively by investors, resulting in an approximately -6\% AR (see \autoref{fig:AR-Ven-1992}).  Here too, pre and post-event CARs are statistically indistinguishable from zero on all days except for the day of the Chavez coup attempt, this time demonstrating the market's negative evaluation of a Pérez to Chavez regime change. These two failed coups therefore demonstrate one positive market reaction to the attempted overthrow of a socialist leader, as well as two examples of negative reactions to an attempted and successful overthrow of pro-market leaders.

\vspace{-0.05cm}
\section{Robustness} \label{subsec: Robustness}

Potential concerns with the results in the \nameref{subsec: Coups}, \nameref{subsec: Assassinations}, and \nameref{subsec: Resignations} sections are that: (1) the ARs could be driven by factors unrelated to the regime changes, (2) the effects of regime changes may be underestimated if investors had apriori information, and (3) confidence intervals based on normally distributed ARs may be inappropriate.

We explore these concerns in two ways. First, we reestimate mean CARs on a set of time-shifted placebo dates, with means computed across all events for each type of regime change. We should not observe significant CARs when performing an identical test on dates where no event occurred. This would call the research design and modeling assumptions into question, and raise concerns that the abnormal returns were caused by factors other than the regime changes. The results reinforce the main results: the ARs on the actual event date capture most of effect of the regime change, although effects can sometimes persist in the short event window following the event date. For a detailed descriptions of this analysis and results, see \nameref{subsec: robustness appendix}.

Second, we create a synthetic control portfolio for each event. Each country is given a weight which represents its influence in the synthetic control portfolio. The weight is chosen so that the daily returns and the variance of the daily returns of the control portfolio and the event country are most similar in the estimation window. The set of possible countries in the control portfolio consists of all countries listed in \autoref{tab:stock-list}.

Third, non-parametric statistical techniques (sign and rank tests) free from distributional assumptions address concerns about inferences from small sample sizes. \autoref{tab:non-parametric} compares event day ARs and the absolute value of all combined returns (since some events increase returns while others decrease them) to the synthetic control portfolio using the non-parametric methods discussed above. The synthetic control and small sample tests suggest that the main results are not a result of deviations from normality or confounding world events (see \nameref{subsec: robustness appendix} for a full discussion). 

% \footnote{See section 8 in \citet{mackinlay1997event} for more details.}

%The ``abnormal absolute returns'' are abnormal returns for the absolute value of stock returns. This is done to combine events since resignations tend to increase returns while assassinations and coups tend to decrease them. The idea that the absolute value of returns might increase during irregular regime changes is similar to the finding that volatility increases and is consistent with \autoref{fig:AV-DR}.

%Both tests are less influenced by departures from normality than statistics based on traditional t-tests such as those reported earlier. 

%We employ the sign and the rank tests, which are based on the sign and the rank of the event day ARs respectively.

\begin{table}[!ht]
\caption{Non-parametric tests of the impact of regime changes} \label{tab:non-parametric}
\vspace{-5pt}
\footnotesize
\begin{center}
\begin{threeparttable}
\begin{tabular*}{\textwidth}{l@{\extracolsep{\fill}}d{1.3}d{1.3}d{1.3}d{1.3}d{1.3}d{1.3}d{1.3}}
  \hline
  \hline
&\multicolumn{3}{c}{Regime Change Country}&\multicolumn{3}{c}{Synthetic Control Portfolio}&\multicolumn{1}{c}{\multirow{2}{*}{Wilcoxon}}\\
\cmidrule(r){2-4} \cmidrule(r){5-7}
&\multicolumn{1}{c}{Mean}&\multicolumn{1}{c}{Rank}&\multicolumn{1}{c}{Sign}
&\multicolumn{1}{c}{Mean}&\multicolumn{1}{c}{Rank}&\multicolumn{1}{c}{Sign}
&\multicolumn{1}{c}{Rank Test}\\
\multicolumn{1}{c}{Event Type}&\multicolumn{1}{c}{CAR (0,0)}&\multicolumn{1}{c}{p-value}&\multicolumn{1}{c}{p-value}
&\multicolumn{1}{c}{CAR (0,0)}&\multicolumn{1}{c}{p-value}&\multicolumn{1}{c}{p-value}
&\multicolumn{1}{c}{p-Value}\\
  \hline
\ExpandableInput{../tables/non-parametric-ar0.txt}
   \hline
   \hline
\end{tabular*}
\scriptsize
Notes: Estimates for assassinations do not include the assassination of U.S. president William McKinley in 1901 because no control portfolios are available.
\end{threeparttable}
\end{center}
\end{table}

\section{Conclusion}

Conventional theory suggests political instability causes capital flight. We show that this is not always the case. Unexpected changes in ruler virtually always increase volatility, but when political instability is broken down into types, evidence emerges that some types of instability increase capital access while other types lead to capital flight. 

%Despite their frequency, there is little evidence on the economic consequences of various types of political instability.  

We examine changes in stock market returns surrounding politically unstable domestic events to show that resignations increase capital access on average, but assassinations and coups usually cause capital flight. We also find evidence that capital markets tend to prefer democratic regime changes to authoritarian shifts, but that even democratically subversive coups can boost capital access if the instigators are clearly pro-business. 

There are a number of avenues for future research. First, more research is needed to identify the pro and anti market characteristics of regime changes. Second, more work is needed to determine the extent to which financial market flows translate to broader economic development outcomes. It remains unclear whether the direction of the effects of different types of regime changes on outcomes such as economic growth, investment, debt, inflation, infant mortality, and years of schooling are consistent with their impact on financial flows, or if investor perceptions are at odds with certain development goals. 

%\theendnotes

%\citet{meyersson2016political} examined the impact of coups on economic growth, investment, debt, inflation, infant mortality, and years of schooling.

%We demonstrate that investors clearly disliked the socialist regime in Venezuela, and \citet{girardi2018institution} show that in Chile increasing stock market valuations were caused by changes in private property rights, but it would be helpful to identify such mechanisms in additional settings.

%The Venezuelan example shows that investors clearly recognized one unstable event would benefit them while the other would harm them. 

%Our results are consistent with the idea that perceptions of government competence and changes in government have large impacts on investor actions. But although the effect of regime changes on stock volatility is substantial, the effect on the direction of stock returns is less certain. 

%Coups and other types of irregular regime changes remain common. Our sample consists of 5 coups, 1 assassination, and 7 resignations since 2000. 

%Despite their frequency, there is little evidence on the economic consequences of various types of political instability. This paper helps fill the evidence gap by using an event study approach that exploits daily returns of national stock market indices. This approach provides well-identified estimates of an economic effect of regime changes that is less susceptible to endogeneity bias than prior cross country studies. The large sample of political events in our study increases the generalizability of our findings.  

%The Arab spring is perhaps the most notable, with protests spreading throughout the Middle East in late 2010. There have also been a number of failed coups such as the 2002 coup attempt against Hugo Chavez and the 2016 failed coup in Turkey.\footnote{The Turkish coup attempt led to negative event-day CARs of approximately -7\%. See Figure A2 for a visual depiction.} 

%Our examination of pre-event trends in abnormal returns suggests that the positive returns we observe following resignations are not driven by investors anticipating resignations, but not coups or assassinations. CARs trend downward in the days preceding resignations, but if investors anticipated a resignation that brought a more competent leader and increased stability, pre-event CARs should also trend positive. 

%These results are consistent with previous research showing that the coup replacing Allende in Chile increased stock market valuations \citep{girardi2018institution}. 

%Such research can help enlarge the body of evidence on the extent to which regime changes cause institutional and political change, and, in turn, have significant consequences for economic development.

\clearpage
\pagebreak

\pdfbookmark[1]{References}{References}
\bibliography{bibliography}

\newpage
\appendix
\setcounter{secnumdepth}{1}
\setcounter{table}{0}
\setcounter{figure}{0}
\renewcommand\thetable{\Alph{section}.\arabic{table}}
\renewcommand\thefigure{\Alph{section}.\arabic{figure}}

\section{Appendix} \label{sec: appendix}

\subsection{Data} \label{subsec: data}

\begin{table}[H]
\caption{List of stock indices} \label{tab:stock-list}
\vspace{-5pt}
\scriptsize
\begin{center}
\begin{threeparttable}
\begin{tabular*}{\textwidth}{l@{\extracolsep{\fill}}lll}
  \hline
  \hline
  \multicolumn{1}{c}{Date}&\multicolumn{1}{c}{Country}&\multicolumn{1}{c}{Begin Date} &\multicolumn{1}{c}{Start Date}\\
  \hline
Argentina & Beunos Aires SE General Index & Dec-66 & Jan-17\\
Australia & Australia ASX All Ordinaries & Jan-58 & Jan-17\\
Bangladesh & Dhaka SE Index & Jan-90 & Jan-17\\
Canada & Canada S\&P/TSX 300 Composite & Jan-76 & Jan-17\\
Chile & Santiago IGBC General Index & Jan-75 & Jan-17\\
Colombia & Colombia IGBC General Index & Jan-92 & Jan-17\\
Ecuador & Ecuador Bolsa de Valores de Guayaquil (BVG) & Jan-94 & Jan-17\\
Egypt & Cairo SE Index & Dec-92 & Jan-17\\
Emerging Market & S\&P/IFC Emerging Markets Investable Composite & Jul-95 & Jan-17\\
Greece & Athens SE General Index & Oct-88 & Jan-17\\
India & Bombday SE SENSEX & Apr-79 & Jan-17\\
Indonesia & Jakarta SE Composite Index & Apr-83 & Jan-17\\
Iran & Tehran SE Price Index & Jan-95 & Jan-17\\
Israel & Tel Aviv 100 Index & May-87 & Jan-17\\
Japan & Tokyo SE Price Index (TOPIX) & Jan-53 & Jan-17\\
Latin America & Dow Jones Latin America Index & Jan-92 & Jan-17\\
Lithuania & Lithuania Lit-10 Index & Jan-99 & May-05\\
Malaysia & Malaysia KLSE Composite & Jan-80 & Jan-17\\
Nepal & Nepal NEPSE Stock Index & Jan-01 & Jan-17\\
Netherlands & Netherlands All-Share Price Index & Jan-80 & Jan-17\\
Pakistan & Karachi SE 100 Index & Jan-89 & Jan-17\\
Paraguay & Asuncion SE PDV General Index & Oct-93 & Sep-08\\
Peru & Lima SE General Index & Jan-82 & Jan-17\\
Philippines & Manila SE Composite Index & Jan-86 & Jan-17\\
Portugal & Oporto PSI-20 Index & Jan-86 & Jan-17\\
Singapore & Singapore FTSE ST Index & Jul-65 & Jan-17\\
South Korea & Korea SE Stock Price Index  & Jan-62 & Jan-17\\
Southeast Asia & Dow Jones Southeast Asia Index & Jan-92 & Jan-17\\
Spain & Madrid SE General Index & Aug-71 & Jan-17\\
Sri Lanka & Colombo SE All-Share Index & Dec-84 & Jan-17\\
Sweden & Sweden OMX Affarsvarlden General Index & Jan-80 & Jan-17\\
Taiwan & Taiwan SE Capitalization Weighted Index & Jan-67 & Jan-17\\
Thailand & Thailand SET General Index & Apr-75 & Jan-17\\
Tunisia & Tunisia SE Index & Dec-97 & Jan-17\\
Turkey & Instanbul IMKB 100 Price Index & Oct-87 & Jan-17\\
Ukraine & Ukraine PFTS OTC Index & Jan-98 & Jan-17\\
United Kingdom & UK FTSE All-Share Index & Nov-62 & Jan-17\\
United States & Dow Jones Industrial Average & Feb-1885 & Jan-17\\
Uruguay & Bolsa de Valores de Montevideo Index & Jan-08 & Jul-16\\
Venezuela & Dow Jones Venezuela Stock Index & Jan-92 & Jul-07\\
Venezuela & Caracas SE General Index & Jan-94 & Jan-17\\
World & MSCI World Price Index & Jan-76 & Jan-17\\
Zambia & Lusaka SE Index & Jan-02 & Apr-06\\
Zambia & Lusaka SE Index & Jul-11 & Jan-17\\
   \hline
   \hline
\end{tabular*}
\end{threeparttable}
\end{center}
\end{table}

\begin{table}[H]
\caption{Regime changes} \label{tab:regime-changes}
\vspace{-5pt}
\scriptsize
\begin{center}
\begin{threeparttable}
\begin{tabular*}{\textwidth}{l@{\extracolsep{\fill}}ll}
  \hline
  \hline
  \multicolumn{1}{c}{Date}&\multicolumn{1}{c}{Country}&\multicolumn{1}{c}{Political Outcome}\\
  \hline
   \multicolumn{3}{c}{\textbf{Coups}}\\ 
\ExpandableInput{../tables/coups.txt} \\
   \multicolumn{3}{c}{\textbf{Failed coup}}\\ 
\ExpandableInput{../tables/failed_coup.txt}\\
  \multicolumn{3}{c}{\textbf{Assassinations}}\\ 
\ExpandableInput{../tables/assassinations.txt}\\
  \multicolumn{3}{c}{\textbf{Resignations}}\\ 
\ExpandableInput{../tables/resignations.txt}
   \hline
   \hline
\end{tabular*}
\scriptsize
Notes: The Polity score is used to classify political outcomes as follows: autocracy = $-10 \leq \rm{score} \leq -6$, anocracy = $-5 \leq \rm{score} \leq 5$, and democracy = $6 \leq \rm{score} \leq 10$.
\end{threeparttable}
\end{center}
\end{table}

\begin{figure}[!ht]
\begin{centering}
\includegraphics[width = 0.6\textwidth]{../figs/daily_mean_absreturn.pdf}
\caption{Absolute value of daily returns}
\label{fig:AV-DR}
\end{centering}
\end{figure}

\begin{figure}[!htb]
\centering
\includegraphics[width = 0.6\textwidth]{../figs/mean-volatility.pdf}
\caption{Mean of volatility estimates from GARCH(1,1) models}
\label{fig:volatility}
\end{figure}

%\clearpage
%\newpage

\subsection{Synthetic control methods} \label{subsec: synth}

Formally, let $\boldsymbol{R}_{k}$ be the vector of returns for the event country in the estimation window, $\boldsymbol{R}_{-k}$ be the vector of returns for all other countries in the estimation window, $\boldsymbol{X}_1=(\boldsymbol{R}_{k},\var(\boldsymbol{R}_{k}))$, $\boldsymbol{X}_0=(\boldsymbol{R}_{-k},\var(\boldsymbol{R}_{-k}))$, and $\boldsymbol{W}_{-k}$ be a $((N-1) \times 1)$ vector of weights where $N$ is the number of countries listed in \autoref{tab:stock-list}. Then $\boldsymbol{W}^*$ is chosen to minimize $(\boldsymbol{X}_1-\boldsymbol{X}_0\boldsymbol{W})'\boldsymbol{V}(\boldsymbol{X}_1-\boldsymbol{X}_0\boldsymbol{W})$ subject to $w_i\geq0$ $(i = 1,2,\ldots,N-1)$ and $\sum_i^{N-1} w_i = 1$, and the vector $\boldsymbol{V}$ is chosen so that stock returns for the control portfolio during the estimation window are are close as possible to the event country.\footnote{See \citet{abadie2003economic} for further details.} 

%\newpage

%\subsection{Volatility} \label{subsec: volatility}

%\newpage

\subsection{Event-level estimates} \label{subsec: coups appendix}

\autoref{tab:AR-coups} shows abnormal returns for national stock indices both preceding and following coup d'etat. \autoref{tab:AR-coups} contains all coups presented in \autoref{tab:regime-changes}, with the exception of the Argentinian coup of March 24, 1976 which is excluded from our analysis because the stock market remained closed for twelve days following the event.\footnote{Treating this twelve day period as a single day CARs results in a positive abnormal return of 58\%, a fluctuation that seems qualitatively unreasonable.} 

 $(0,\tau-1)$ denotes the $\tau$-day period beginning with the event day and $(-1,\tau)$ denotes the negative $\tau$-day period beginning with the day prior to the event day. We present results for the sum of abnormal returns over the post-event windows of the event date only $(0,0$), the event date plus 6 days $(0,6$) and 19 days $(0,19$), and the pre-event windows $(-1,-7$) and $(-1,-20$).

\begin{table}[H]
\caption{Abnormal returns following coups} \label{tab:AR-coups}
\vspace{-5pt}
\scriptsize
\begin{center}
\begin{threeparttable}
\begin{tabular*}{\textwidth}{l@{\extracolsep{\fill}}ld{4}d{4}d{4}d{4}d{4}d{2}}
  \hline
  \hline
\multicolumn{2}{c}{} & \multicolumn{3}{c}{Post-Event CAR} & \multicolumn{2}{c}{Pre-Event CAR} & \multicolumn{1}{c}{\multirow{2}{*}{Days to}}\\
\cmidrule(r){3-5} \cmidrule(r){6-7}
\multicolumn{1}{c}{Country} & \multicolumn{1}{c}{Event Date} & \multicolumn{1}{c}{(0,0)} & \multicolumn{1}{c}{(0,6)} & \multicolumn{1}{c}{(0,19)} & \multicolumn{1}{c}{(-1,-7)} & \multicolumn{1}{c}{(-1,-20)} & \multicolumn{1}{c}{rebound}\\
  \hline
\ExpandableInput{../tables/artable-coups-car.txt}
  \hline
\ExpandableInput{../tables/artable-coups-car-mean.txt}
   \hline
   \hline
\end{tabular*}
\scriptsize
Notes: Standard errors are in parentheses. ``Days to rebound'' is the number of trading days following a negative stock return for the national stock index to return to pre-event level (it is calculated if the price decreases on the event day, not if the event day abnormal return is negative). Returns are inflation adjusted. 
\end{threeparttable}
\end{center}
\end{table}

%\subsection{Event-level estimates: assassinations} \label{subsec: ass appendix}

\begin{table}[H]
\caption{Abnormal returns following assassinations} \label{tab:AR-ass}
\vspace{-5pt}
\scriptsize
\begin{center}
\begin{threeparttable}
\begin{tabular*}{\textwidth}{l@{\extracolsep{\fill}}ld{4}d{4}d{4}d{4}d{4}d{2}}
  \hline
  \hline
\multicolumn{2}{c}{} & \multicolumn{3}{c}{Post-Event CAR} & \multicolumn{2}{c}{Pre-Event CAR} & \multicolumn{1}{c}{\multirow{2}{*}{Days to}}\\
\cmidrule(r){3-5} \cmidrule(r){6-7}
\multicolumn{1}{c}{Country} & \multicolumn{1}{c}{Event Date} & \multicolumn{1}{c}{(0,0)} & \multicolumn{1}{c}{(0,6)} & \multicolumn{1}{c}{(0,19)} & \multicolumn{1}{c}{(-1,-7)} & \multicolumn{1}{c}{(-1,-20)} & \multicolumn{1}{c}{rebound}\\
  \hline
\ExpandableInput{../tables/artable-ass-car.txt}
  \hline
\ExpandableInput{../tables/artable-ass-car-mean.txt}
   \hline
   \hline
\end{tabular*}
\scriptsize
Notes: Standard errors are in parentheses. ``Days to rebound'' is the number of trading days following a negative stock return for the national stock index to return to pre-event level (it is calculated if the price decreases on the event day, not if the event day abnormal return is negative). Returns are inflation adjusted. 
\end{threeparttable}
\end{center}
\end{table}

%\subsection{Event-level estimates: resignations} \label{subsec: resignation appendix}

\begin{table}[H]
\caption{Abnormal returns following resignations} \label{tab:AR-resignations}
\vspace{-5pt}
\scriptsize
\begin{center}
\begin{threeparttable}
\begin{tabular*}{\textwidth}{l@{\extracolsep{\fill}}ld{4}d{4}d{4}d{4}d{4}d{2}}
  \hline
  \hline
\multicolumn{2}{c}{} & \multicolumn{3}{c}{Post-Event CAR} & \multicolumn{2}{c}{Pre-Event CAR} & \multicolumn{1}{c}{\multirow{2}{*}{Days to}}\\
\cmidrule(r){3-5} \cmidrule(r){6-7}
\multicolumn{1}{c}{Country} & \multicolumn{1}{c}{Event Date} & \multicolumn{1}{c}{(0,0)} & \multicolumn{1}{c}{(0,6)} & \multicolumn{1}{c}{(0,19)} & \multicolumn{1}{c}{(-1,-7)} & \multicolumn{1}{c}{(-1,-20)} & \multicolumn{1}{c}{rebound}\\
  \hline
\ExpandableInput{../tables/artable-res-car.txt}
  \hline
\ExpandableInput{../tables/artable-res-car-mean.txt}
   \hline
   \hline
\end{tabular*}
\scriptsize
Notes: Standard errors are in parentheses. ``Days to rebound'' is the number of trading days following a negative stock return for the national stock index to return to pre-event level (it is calculated if the price decreases on the event day, not if the event day abnormal return is negative). Returns are inflation adjusted. 
\end{threeparttable}
\end{center}
\end{table}

%\newpage

\subsection{Public Protests} \label{subsec: Public Protests}

%The resignations studied in this paper are those in which leaders left office because of poor performance, public discontent and popular protests. It is therefore not unreasonable to expect the political actions preceding the resignations to have similarly large effects on financial markets. To examine this, 

We explore all resignations that were driven by significant popular demonstrations, riots, non-violent civil resistance and other forms of public discontent (see \autoref{tab:protest-list}).\footnote{The set of resignations includes all those listed in either the Coup d'etat Events Handbook or the Archigos Version 4.1 data set with available financial data. In practice, this is the 2011 Egyptian Revolution and the list of resignations in \autoref{tab:AR-resignations}.}

\begin{table}[!htb]
\caption{List of public protests preceding resignations} \label{tab:protest-list}
\vspace{-5pt}
\scriptsize
\begin{center}
\begin{threeparttable}
\begin{tabular*}{\textwidth}{l@{\extracolsep{\fill}}llll}
  \hline
    \hline
\multicolumn{1}{c}{Country}&\multicolumn{1}{c}{Name}&\multicolumn{1}{c}{Start Date}&\multicolumn{1}{c}{End Date}\\
  \hline
Philippines & EDSA 1/Yellow Revolution & 2/22/1986 & 2/25/1986\\
Bangladesh & Bangladeshi Spring of 1990 & 11/27/1990 & 12/7/1990\\
Thailand & Black May & 5/17/1992 & 5/20/1992\\
Indonesia & Indonesian Riots & 5/12/1998 & 5/21/1998\\
Philippines & EDSA II & 1/17/2001 & 1/20/2001\\
Argentina & Argentina Riots & 12/16/2001 & 12/20/2001\\
Ukraine & Orange Revolution & 11/22/2004 & 1/23/2005\\
Ecuador & Ecuadorian Protests & 4/13/2005 & 4/20/2005\\
Nepal & Nepalese People's Revolution & 4/6/2006 & 4/24/2006\\
Tunisia & Tunisian Revolution & 12/18/2010 & 1/14/2011\\
Egypt & Egyptian Revolution & 1/25/2011 & 2/11/2001\\
   \hline
   \hline
\end{tabular*}
\scriptsize
\end{threeparttable}
\end{center}
\end{table}

%A recent example of a popular uprising preceding a resignation is the 2011 Egyptian Revolution that resulted in the overthrow of President Hosni Mubarak's regime.\footnote{Abnormal returns for this event are not shown in \autoref{tab:AR-resignations} because the stock market was closed on the day of Mubarak's resignation.} Clashes between security forces and protestors led to the deaths of hundreds of citizens and injuries to thousands more. The uprising began on January 25, 2011 when millions of protestors demanded the overthrow of the Egyptian leadership. Examples of public discontent included demonstrations, marches, riots, non-violent civil disobedience, and labor strikes.
%
%The short-term impact of the Egyptian Revolution on the economy was disastrous. As shown in \autoref{fig:CAR-Egypt}, abnormal returns on the Egyptian Stock Exchange Index (EGX 30) were around -7\% on January 26th and -10\% the day after. To prevent further decline during the uprising, the Egyptian Stock Exchange closed at the end of trading on January 27th. President Mubarak resigned on February 11, but the market remained closed until March 23, when CARs declined by another 9\%, before rebounding slightly thereafter.
%
%\begin{figure}[!ht]
%\centering
%\includegraphics[width = 0.6\textwidth]{../figs/egypt-revolution-2011.pdf}
%\caption{Cumulative abnormal returns during the Egyptian revolution}
%\label{fig:CAR-Egypt}
%\end{figure}
%
%An important question is whether other popular uprisings have had similar adverse economic consequences. To examine this, we explore all resignations that were driven by significant public protests.\footnote{The set of resignations includes all those listed in either the Coup d'etat Events Handbook or the Archigos Version 4.1 data set with available financial data. In practice, this is the 2011 Egyptian Revolution and the list or resignations in \autoref{tab:AR-resignations}.} Public protests include popular demonstrations, riots, non-violent civil resistance and other forms of public discontent. We find that both volatility and the absolute value of returns increase during times of protest. Similarly to coups, however, the direction of returns is dependent upon the nature of the protest in question. 

The start and end dates in \autoref{tab:protest-stocks} are the dates that protests began and leader's resigned respectively. Resignations caused by popular uprisings were identified by examining the descriptions in the Coup d'etat Events Handbook and Archigos Version 4.1. Additional Lexis Nexis searches were used to verify these descriptions.

In \autoref{tab:protest-stocks}, we examine whether public protests influence stock prices. The variable \textit{Protest} is equal to 1 during the dates in which citizens participate in political activities demanding the resignation of the executive and 0 otherwise. Non-protest dates are the 250 days prior to the start dates and after the end dates listed in Table A.1.\footnote{The volatility estimates used as the dependent variable in column (4) are estimated on the 250 days prior to the start date, the protest dates, and the 250 days following the end date.}

Column (1) suggests that public protests have no effect on stock returns. However, this occurs because some political movements increase stock prices while others decrease them. As shown in column (2), the absolute value of stock returns are approximately 1.5\% higher during public protests. These estimates would be biased if protest dates are correlated with higher world or regional stock market indices. To address this potential confounder, column (3) controls for returns on the S\&P/IFC Emerging Markets Investable Composite Stock Index. The coefficient on \textit{Protest} barely changes and the absolute value of returns are still about 1.5\% higher during public protests. Finally, column (4) shows that stock volatility is approximately 1 percentage point higher during political movements.

%\footnote{Volatility estimation methodology is described in detail in \nameref{subsec: Volatility}.}
 

We therefore find that both volatility and the absolute value of returns increase during times of protest. Similarly to coups, however, the direction of returns is dependent upon the nature of the protest in question. 

\begin{table}[!htb]
\caption{Effect of public protests on stock prices} \label{tab:protest-stocks}
\vspace{-5pt}
\scriptsize
\begin{center}
\begin{threeparttable}
\begin{tabular*}{\textwidth}{l@{\extracolsep{\fill}}d{1.3}d{1.3}d{1.3}d{1.3}}
  \hline
  \hline
\multicolumn{1}{c}{}&\multicolumn{1}{c}{Returns} & \multicolumn{2}{c}{Absolute Value of Returns}&\multicolumn{1}{c}{Volatility}\\
\cmidrule(r){2-2} \cmidrule(r){3-4} \cmidrule(r){5-5}
 & \multicolumn{1}{c}{(1)}&\multicolumn{1}{c}{(2)}&\multicolumn{1}{c}{(3)}&\multicolumn{1}{c}{(4)}\\
  \hline
\ExpandableInput{../tables/protest-regtable.txt}
   \hline
   \hline
\end{tabular*}
\scriptsize
Notes: Standard errors clustered by event are in parentheses.
\end{threeparttable}
\end{center}
\end{table}

%\clearpage
%\pagebreak
%\newpage

%\subsection{Event-level estimates: democratic vs. authoritarian regime changes}

\begin{table}[!htb]
\caption{Abnormal returns following authoritarian regime changes} \label{tab:AR-auth}
\vspace{-5pt}
\scriptsize
\begin{center}
\begin{threeparttable}
\begin{tabular*}{\textwidth}{l@{\extracolsep{\fill}}ld{4}d{4}d{4}d{4}d{4}d{2}}
  \hline
  \hline
\multicolumn{2}{c}{} & \multicolumn{3}{c}{Post-Event CAR} & \multicolumn{2}{c}{Pre-Event CAR} & \multicolumn{1}{c}{\multirow{2}{*}{Days to}}\\
\cmidrule(r){3-5} \cmidrule(r){6-7}
\multicolumn{1}{c}{Country} & \multicolumn{1}{c}{Event Date} & \multicolumn{1}{c}{(0,0)} & \multicolumn{1}{c}{(0,6)} & \multicolumn{1}{c}{(0,19)} & \multicolumn{1}{c}{(-1,-7)} & \multicolumn{1}{c}{(-1,-20)} & \multicolumn{1}{c}{rebound}\\
  \hline
\ExpandableInput{../tables/artable-auth-car.txt}
  \hline
\ExpandableInput{../tables/artable-auth-car-mean.txt}
   \hline
   \hline
\end{tabular*}
\scriptsize
Notes: Standard errors are in parentheses. ``Days to rebound'' is the number of trading days following a negative stock return for the national stock index to return to pre-event level (it is calculated if the price decreases on the event day, not if the event day abnormal return is negative). Returns are inflation adjusted. 
\end{threeparttable}
\end{center}
\end{table}

\begin{table}[!htb]
\caption{Abnormal returns following democratic regime changes} \label{tab:AR-dem}
\vspace{-5pt}
\scriptsize
\begin{center}
\begin{threeparttable}
\begin{tabular*}{\textwidth}{l@{\extracolsep{\fill}}ld{4}d{4}d{4}d{4}d{4}d{2}}
  \hline
  \hline
\multicolumn{2}{c}{} & \multicolumn{3}{c}{Post-Event CAR} & \multicolumn{2}{c}{Pre-Event CAR} & \multicolumn{1}{c}{\multirow{2}{*}{Days to}}\\
\cmidrule(r){3-5} \cmidrule(r){6-7}
\multicolumn{1}{c}{Country} & \multicolumn{1}{c}{Event Date} & \multicolumn{1}{c}{(0,0)} & \multicolumn{1}{c}{(0,6)} & \multicolumn{1}{c}{(0,19)} & \multicolumn{1}{c}{(-1,-7)} & \multicolumn{1}{c}{(-1,-20)} & \multicolumn{1}{c}{rebound}\\
  \hline
\ExpandableInput{../tables/artable-dem-car.txt}
  \hline
\ExpandableInput{../tables/artable-dem-car-mean.txt}
   \hline
   \hline
\end{tabular*}
\scriptsize
Notes: Standard errors are in parentheses. ``Days to rebound'' is the number of trading days following a negative stock return for the national stock index to return to pre-event level (it is calculated if the price decreases on the event day, not if the event day abnormal return is negative). Returns are inflation adjusted. 
\end{threeparttable}
\end{center}
\end{table}

%\clearpage
%\pagebreak
%\newpage


\subsection{Robustness details} \label{subsec: robustness appendix}

We shift event dates surrounding the actual event date backwards and forwards in increments of five days (-20, 15, 10, 5, 0, 5, 10, 15, and 20 days) (shown in \autoref{fig:mean-car-by-regime-change}). In addition, we extend the forward shifted event dates to one year (110, 195, 285, and 365 days) to capture dates that are likely to be completely unaffected by the regime change. 

\autoref{subfig:mean-car-by-regime-change-placebo} shows CARs estimated with the event date shifted 1-year (365 days) into the future. In contrast to \autoref{fig:mean-car-by-regime-change}, there are no discernible abnormal returns in \autoref{subfig:mean-car-by-regime-change-placebo}. To ensure that results of the placebo test based on 1-year did not occur by chance, \autoref{fig:mean-event-day-ar-by-regime-change} plots event-day abnormal against the number of days shifted. There are a few instances in which there are statistically significant ARs in the same direction as the event day, but they are always smaller in magnitude than the ARs estimated using the actual event date and occur when dates are shifted within the post-event window (days 0-20), a period during which stock returns remain volatile and, in the case of resignations, there is a consistent upward trend in the CARs.  Overall, these results reinforce the main results: the ARs on the actual event date capture most of effect of the regime change, although effects can sometimes persist in the short event window following the event date. These figures also provide evidence that regime changes appear to be unexpected. For instance, the CARs prior to assassinations and coups presented in \autoref{fig:mean-car-by-regime-change} tend to be close to zero, suggesting that investors were unaware that a negative event was likely to occur in the coming days. For resignations, CARs trend downward in the 10 days prior to the event; however, if investors were aware that a resignation were about to occur one would expect the pre-event CARs to be positive given the positive CARs observed in the post-event window. There is thus little evidence that the CARs in the post-event window are not capturing most of the effect caused by the regime changes.

There are a few instances in which there are statistically significant (at the 5\% level) ARs in the same direction as those on the actual event day, but they are always smaller in magnitude than the ARs estimated using the actual event date and most occur when dates are shifted within the post-event window (days 0-20), a period during which stock returns remain volatile and, in the case of resignations, there is a consistent upward trend in the CARs. When dates are shifted forward further ($\geq$ 110 days), the event day AR is only statistically significant in one case (day 365 for assassinations). The event day ARs are considerably smaller in magnitude and are not statistically different from zero for either coups or resignations. Moreover, while CARs estimated using the actual event date for resignations trend upwards following the event day, there is no consistent trend in the placebo analysis. \autoref{fig:mean-car-by-regime-change} therefore suggests that the main results are not merely an artifact of the data.  

\begin{figure}[!htb]
\centering
\includegraphics[max size={0.75\textwidth}]{../figs/mean-car-by-regime-change-type-placebo.pdf}
\caption{Event date shifted forward by one year} \label{subfig:mean-car-by-regime-change-placebo}
\end{figure}

\begin{figure}[!htb]
\centering
\includegraphics[width = 0.75\textwidth]{../figs/mean-ar-by-regime-change-type-placebo.pdf}
\caption{Time-shifted placebo sensitivity analysis of mean event day abnormal return by type of regime change}
\label{fig:mean-event-day-ar-by-regime-change}
\end{figure}

As shown in \autoref{tab:non-parametric}, the mean event day abnormal returns for coups, assassinations and resignations are all statistically different from zero at the 1\% level using the rank test statistic and the abnormal returns for coups and assassinations are significant at at least the 10\% level using the sign test. In addition, abnormal absolute returns for all events are statistically significant at at least the 5\% level using both the rank and sign tests. On the other hand, the event day abnormal returns for the control portfolio are never statistically different from zero at even the 10\% level using the rank or sign tests. Finally, the difference in means between the regime change country and the control portfolio are statistically different from zero for coups (1\% level), assassinations (10\% level), resignations (5\% level), and all events combined (1\% level) when using two-sided p-values from the Wilcoxon rank test.\footnote{The Wilcoxon rank test is a non-parametric statistical technique that can be used to compare differences between matched samples.} 
%
%\clearpage
%\pagebreak

%\subsection{Graphical depictions of additional events}

%\begin{figure}[!htb]
%\centering
%\vspace{-0.4cm}
%\includegraphics[scale=0.75]{../figs/venezuela_coup_attempt_1992.pdf}
%\caption{Abnormal returns surrounding the 1992 Venezuelan coup attempt}
%\label{fig:AR-Ven-1992}
%\end{figure}


%\begin{figure}[!htb]
%\centering
%\includegraphics[scale=0.75]{../figs/turkey_coup_attempt_2016.pdf}
%\caption{Abnormal returns surrounding the 2016 Turkish coup attempt}
%\label{fig:AR-Turkey-2016}
%\end{figure}


%\pagebreak
%
%\subsection{Synthetic Control Portfolio}
%Let $\boldsymbol{R}_{k}$ be the vector of returns for the event country in the estimation window, $\boldsymbol{R}_{-k}$ be the vector of returns for all other countries in the estimation window, $\boldsymbol{X}_1=(\boldsymbol{R}_{k},\var(\boldsymbol{R}_{k}))$, $\boldsymbol{X}_0=(\boldsymbol{R}_{-k},\var(\boldsymbol{R}_{-k}))$, and $\boldsymbol{W}_{-k}$ be a $((N-1) \times 1)$ vector of weights where $N$ is the number of countries listed in \autoref{tab:stock-list}. Then $\boldsymbol{W}^*$ is chosen to minimize $(\boldsymbol{X}_1-\boldsymbol{X}_0\boldsymbol{W})'\boldsymbol{V}(\boldsymbol{X}_1-\boldsymbol{X}_0\boldsymbol{W})$ subject to $w_i\geq0$ $(i = 1,2,\ldots,N-1)$ and $\sum_i^{N-1} w_i = 1$, and the vector $\boldsymbol{V}$ is chosen so that stock returns for the control portfolio during the estimation window are are close as possible to the event country.\footnote{See \citet{abadie2003economic} for further details.}


\end{document} 