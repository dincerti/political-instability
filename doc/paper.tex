\documentclass[12pt,final,fleqn]{article}

% basic packages
\usepackage[margin=1in] { geometry }
\usepackage{amssymb,amsmath, bm}
\usepackage{verbatim}
\usepackage[latin1]{inputenc}
%\usepackage[OT1]{fontenc}
\usepackage{setspace}
\usepackage{enumitem}
\usepackage{url}
\usepackage[font={bf}]{caption}
%\usepackage{pgfplots}
%\usepackage[font={bf}]{caption}
\usepackage{setspace}
\usepackage{latexsym}
%\usepackage{euscript}
\usepackage{graphicx}
\usepackage{marvosym}
%\usepackage[varg]{txfonts}  Older version of ``g'' in math.

% bibliography packages
\usepackage[natbibapa]{apacite}
\bibliographystyle{apacite}
\bibpunct{(}{)}{;}{a}{}{,}
\renewcommand{\bibname}{References}

% hyperref options
\usepackage{color}
\usepackage{hyperref}
\usepackage{xcolor}
\hypersetup{
    colorlinks,
    linkcolor={blue!50!black},
    citecolor={blue!50!black},
    urlcolor={blue!80!black}
}
\newcommand*{\Appendixautorefname}{Appendix}
\renewcommand*{\sectionautorefname}{Section}
\renewcommand*{\subsectionautorefname}{Section}
\newcommand{\aref}[1]{\hyperref[#1]{Appendix~\ref{#1}}}

% packages for tables
\usepackage{longtable}
\usepackage{booktabs, threeparttable}
\usepackage{threeparttablex}
%\usepackage{tabularx}
% dcolumn package
\usepackage{dcolumn}
\newcolumntype{.}{D{.}{.}{-1}}
\newcolumntype{d}[1]{D{.}{.}{#1}}
\captionsetup{belowskip=10pt,aboveskip=-5pt}
\usepackage{multirow}
% rotating package
\usepackage[figuresright]{rotating}
\usepackage{pdflscape}
\usepackage{subcaption}

% packages for figures
\usepackage{grffile}
\usepackage{afterpage}
\usepackage{float}
\usepackage[section]{placeins}
\usepackage[export]{adjustbox}

% theorem package
\usepackage{theorem}
\theoremstyle{plain}
\theoremheaderfont{\scshape}
\newtheorem{theorem}{Theorem}
\newtheorem{algorithm}{Algorithm}
\newtheorem{assumption}{Assumption}
\newtheorem{lemma}{Lemma}
\newtheorem{proposition}{Proposition}
\newtheorem{remark}{Remark}
\newcommand{\qed}{\hfill \ensuremath{\Box}}
\newcommand\indep{\protect\mathpalette{\protect\independenT}{\perp}}
\DeclareMathOperator{\sgn}{sgn}
\DeclareMathOperator{\tr}{tr}
\DeclareMathOperator{\argmin}{arg\min}
\DeclareMathOperator{\argmax}{arg\max}
\def\independenT#1#2{\mathrel{\rlap{$#1#2$}\mkern2mu{#1#2}}}
\providecommand{\norm}[1]{\lVert#1\rVert}
\renewcommand\r{\right}
\renewcommand\l{\left}
\newcommand\E{\mathbb{E}}
\newcommand\dist{\buildrel\rm d\over\sim}
\newcommand\iid{\stackrel{\rm i.i.d.}{\sim}}
\newcommand\ind{\stackrel{\rm indep.}{\sim}}
\newcommand\cov{{\rm Cov}}
\newcommand\var{{\rm Var}}
\newcommand\SD{{\rm SD}}
\newcommand\bone{\mathbf{1}}
\newcommand\bzero{\mathbf{0}}

% dotted lines in tables
%\usepackage{arydshln}

\usepackage{pdflscape}

% spacing between sections and subsections
\usepackage[compact]{titlesec}

% times new roman
%\usepackage{times}

% appendix settings
\usepackage[toc,page,header]{appendix}
\renewcommand{\appendixpagename}{\centering Appendices}
\usepackage{chngcntr}
\usepackage{etoolbox}
\usepackage{lipsum}


% file paths and definitions
\makeatletter
\newcommand*\ExpandableInput[1]{\@@input#1 }
\makeatother

\setlength{\mathindent}{1cm}
\allowdisplaybreaks[4]
\doublespacing
%\special{pdf: pagesize width 8.5truein height 11.0truein}

\titleformat{\subsection}
  {\itshape\large}{\thesubsection}{1em}{}

\setcounter{tocdepth}{1}
\begin{document}
\author{Devin Incerti and Trevor Incerti}
\title{\textbf{Are regime changes always bad economics? Evidence from daily financial data}}
\date{\today}
\maketitle

\singlespacing
\begin{abstract}
We examine the relationship between political instability and daily returns of national stock indices in every country that experienced a coup, assassination, or resignation for which daily data is available. A common hypothesis is that political instability discourages investment and reduces economic growth. By contrast, we find that while financial volatility increases dramatically following ``irregular'' regime changes, different types of regime change have disparate effects on stock returns. We use daily financial data and a constant mean return event study model to show that abnormal returns following resignations are large and positive (4\%), while abnormal returns following assassinations are negative and smaller in magnitude (2\%). The impact of coups tends to be negative (2\%), but some events result in positive abnormal returns of 10\% or more. We also find that volatility increases during times of protest preceding resignations, but that no clear directionality is present. We therefore find that the expected direction and magnitude of abnormal returns is dependent on the type of political event and its expected impact on economic policy.
\end{abstract}
\doublespacing

\section{Introduction} \label{sec:Introduction}

The image of a coup or popular uprising ousting an autocratic ruler and installing a democratic government is an evocative one. However, both recent events and prior research suggest that coups rarely lead to democracy. In the wake of the Arab Spring---in which rulers were deposed in four countries---only Tunisia ultimately transitioned to democracy. These results are in line with prior research that claims that in autocracies coups rarely lead to democratization, and in democracies they cause instability \citep{derpanopoulos2015coups, marinov2014coups, powell2011global, thyne2016coup, varol2011democratic}.

Of course, coups and other regime changes not only impact a country's political environment but its economy as well. For instance, investors, multinational firms, development agencies, and aid organizations rank political risk as a top consideration when making investment decisions in emerging markets.\footnote{In the 2013 Multilateral Investment Guarantee Agency (MIGA) \citet{wipr2013} report (the last report published), executives of multinational enterprises (MNEs) ranked political risk as the second most important constraint for foreign direct investment (FDI) in developing countries over the next three years (after macroeconomic instability). In \citet{wipr2011} and  \citet{wipr2012}, political risk was ranked the most important constraint for FDI in developing countries - even greater than macroeconomic instability. The types of political risk of most concern to investors in developing countries (ranked in order of importance) were 1) adverse regulatory changes, 2) breaches of contract, 3) transfers and convertibility restrictions, 4) civil disturbances, 5) non-honoring of government guarantees, 6) expropriation nationalization, 7) terrorism and 8) war.} Common rationales are that unstable governments act myopically and adopt inefficient policies \citep{svensson1998investment,devereux1998political}\footnote{\citet{svensson1998investment} argues that governments are less likely to invest in the legal system and the protection of property rights when political uncertainty is high. Similarly, \citet{devereux1998political} propose a model in which political instability causes governments to leave fewer assets to their successors which forces them into increasing capital taxes. The knowledge of future taxation then causes the private sector to reduce current investment which reduces future output.} and that policy uncertainty depresses business activity \citep{baker2016measuring}. Such sentiments are supported by empirical research that suggests political instability---as measured by regime change frequency or political violence---is negatively correlated with investment, financial development, and GDP growth in cross country regressions \citep{aisen2013does, alesina1996income, alesina1996political, fosu1992political, jong2009measurement, roe2011political} and that the variance of stock market returns increases in response to economic policy uncertainty \citep{leblang2005government, jensen2005market, liu2015economic}.

But such hypotheses are often overly simplistic. Not all political instability is equivalent - resignations, assassinations, and coups are all examples of regime changes, but may have disparate effects on financial returns. Resignations typically occur when an unpopular leader steps down and may signal the coming of a more effective leader. Assassinations are not always related to the effectiveness of a leader, can occur seemingly at random, and the successor to the assassinated leader may be unclear. Coups can occur in the name of democracy or autocracy, and in general, signal relative weakness in the current political system. Further, political instability may be seen as a positive event if the current regime is regarded as strongly anti-business or anti-global.  \citet{alesina1996political} take the possibility of a new government adopting better economic policies seriously, but argue that the negative effects of uncertainty dominate the positive effects of coups staged by pro growth factions. Likewise, \citet{londregan1990poverty} find that coups have no impact on economic growth and hypothesize that this may because there is a bimodal distribution of coups, some of which enhance growth and others which restrict it. More recent studies provide support for this hypothesis. \citet{meyersson2016political} finds that coup attempts that occur in democratic regimes are associated with negative GDP growth, while those that occur in autocratic regimes are associated with positive GDP growth. \citet{girardi2018institution} show that Salvador Allende's socialist government was associated with large declines in stock prices while the coup that replaced him boosted them.  

In this study, we examine the impact of ``irregular'' regime changes and public protests on economic performance and remain agnostic about the direction of the effect. In contrast to the aforementioned cross country studies but like \citet{girardi2018institution} we  conduct our analysis using daily financial data. Market expectations in the form of stock market returns are used as an indicator of whether the market views different types of potentially destabilizing political events as ``good'' for economic growth. The advantage of this event study approach is that it allows us to use time-series variation in stock returns to estimate treatment effects and mitigate the endogeneity problems that are nearly ubiquitous in cross country regressions.\footnote{A growing body of work uses event studies to assess political phenomenon.  For example, political events have been used to estimate the effect of political connections on firm value \citep[e.g.][]{fisman2001estimating,faccio2006politically,goldman2009politically}. Studies in ``forensic economics" have used abnormal returns from political events to locate or examine transactions such as insider trading \citep{dube2011coups}, illegal arms trading \citep{dellavigna2010detecting}, and the impact of hostilities on the financial performance of diamond mining firms in Angola \citep{guidolin2007diamonds}.} We analyze 13 coups, 8 assassinations, 15 forced resignations, 11 public protests, and 1 natural experiment provided by the failed 2002 coup in Venezuela. Our study is therefore not only able to use an event-study approach to enhance the reliability of the estimates, but is not limited to a single case study as in \citet{girardi2018institution}. The relatively large sample enhances external validity and allows us to show that different types of political events have disparate economic effects. 

We find that coups, assassinations, resignations, and public protests cause large increases in financial volatility. Abnormal returns following resignations are large and positive (4\%) while abnormal returns following assassinations are negative and smaller in magnitude (2\%). The average impact of coups is negative (2\%); however, stock prices increased dramatically during the failed overthrow of Hugo Chavez's socialist regime and decreased by approximately the same magnitude following his return to leadership. The effect of public protests on mean stock returns is uncertain, with some protests such as those during the Egpytian Revolution depressing them and others have the opposite effect. 

These findings are consistent with research that shows that ``good coups''---coups that lead to democratization or economic liberalization---are not the norm and therefore on average coups should not be expected to lead to increased economic development and positive market returns \citep{derpanopoulos2015coups, marinov2014coups, powell2011global, thyne2016coup, varol2011democratic}. However, while ``good coups'' may be the minority, regime changes that attempt to overthrow anti-business autocrats may be expected to result in positive abnormal returns as investors see little risk of a replacement government ``worse'' than the status quo.

Our main conclusion is therefore that the type of political event and its expected impact on economic policy determines the direction of abnormal returns. Events expected to lead to more stable governance, economic liberalization, or democratization (such as willful resignations and coups that overthrow protectionist or leftist autocrats) are associated with positive returns, while those that consolidate authoritarian rule, exacerbate poor economic policies, or merely increase policy uncertainty have the opposite effect.

\section{Data}
Political data are primarily drawn from the Center for Systemic Peace's (CSP) Polity IV Coup d'etat dataset and Coup d'etat Events handbook. The Coup d'etat dataset includes the date of 1) successful coups, 2) attempted coups, 3) plotted coups and 4) alleged coup plots. We focus on successful coups because it is difficult to classify failed coups.\footnote{\citet[p. 617]{needler1966political} has even gone so far as to say that ``the categories of coups that were aborted, suppressed, or abandoned melt into each other and into a host of other non-coup phenomena so as to defy accounting.''} The one exception is that we separately study the failed coup attempt in Venezuela in April 2002 in which the President of Venezuela, Hugo Chavez, was removed from office for two days because it provides a natural experiment that allows us to estimate both the impact of the seemingly successful removal of a left-wing populist with a pro-business regime, and the ensuing reinstatement of left-wing populist. 

\begin{table}[!ht]
\caption{Regime changes} \label{tab:regime-changes}
\vspace{-5pt}
\footnotesize
\begin{center}
\begin{threeparttable}
\begin{tabular*}{\textwidth}{l@{\extracolsep{\fill}}ll}
  \hline
  \hline
  \multicolumn{1}{c}{Date}&\multicolumn{1}{c}{Country}&\multicolumn{1}{c}{Political Outcome}\\
  \hline
   \multicolumn{3}{c}{\textbf{Coups}}\\ 
\ExpandableInput{../tables/coups.txt} \\
   \multicolumn{3}{c}{\textbf{Failed coup}}\\ 
\ExpandableInput{../tables/failed_coup.txt}\\
  \multicolumn{3}{c}{\textbf{Assassinations}}\\ 
\ExpandableInput{../tables/assassinations.txt}\\
  \multicolumn{3}{c}{\textbf{Resignations}}\\ 
\ExpandableInput{../tables/resignations.txt}
   \hline
   \hline
\end{tabular*}
\scriptsize
Notes: The Polity score is used to classify political outcomes as follows: autocracy = $-10 \leq \rm{score} \leq -6$, anocracy = $-5 \leq \rm{score} \leq 5$, and democracy = $6 \leq \rm{score} \leq 10$.
\end{threeparttable}
\end{center}
\end{table}

The Coup d'etat Events handbook also provides a list of 1) auto-coups\footnote{Defined by Center for Systemic Peace as the ``occurrence of subversion of the constitutional order by a ruling (usually elected) executive and the imposition of an autocratic regime."}, 2) the ouster of leadership by foreign forces, 3) the ouster of leadership by rebel forces, 4) assassinations of the executive and 5) resignations of the executive due to poor performance and/or loss of authority. Daily financial data is available for countries in categories 4 and 5, so we supplement the coups with assassinations and resignations to form a list of ``irregular'' regime changes. The resignations are those in which the ruling executive was coerced to resign due to poor performance, public discontent and popular demonstrations.

We further supplement the CSP data with leadership data from Archigos Version 4.1, which allows us to identify additional cases in which a ``leader lost power through irregular means.'' Irregular transfers of power are those in which leaders do not leave office ``in a manner prescribed by either explicit rules or established conventions.'' Nearly all removals by irregular means result from the threat or use of force (e.g. coups, revolts and assassinations). 

A list of the political events in our dataset is shown in \autoref{tab:regime-changes}. Coups tend to have the largest impact on the level of democratization as a number of countries have subeqeuently transitioned from democracies to anocracies or autocracies. On the other hand, neither assassinations or resignations have much of an impact on the level of democratization. 

Financial data are from the Global Financial Data database, which includes the longest available time series of stock prices. We collect data on national equity indices and two global equity indices, the S\&P/IFC Emerging Market Investable Composite and the Morgan Stanley Capital International (MSCI) World Index. The S\&P/IFC index includes securities from emerging markets while the MSCI index includes securities from developed markets only. We collect stock index data on every country in which there was a coup or coup attempt, an assassination or failed assassination, or a forced resignation.\footnote{The list of failed assassinations are from \citet{jones2009hit}. Coup attempts are those in category 2 in the CSP Coup d'etat dataset.} The longest available daily time series for these stock indices are listed in \autoref{tab:stock-list}.

\begin{table}[!ht]
\caption{List of stock indices} \label{tab:stock-list}
\vspace{-5pt}
\footnotesize
\begin{center}
\begin{threeparttable}
\begin{tabular*}{\textwidth}{l@{\extracolsep{\fill}}lll}
  \hline
  \hline
  \multicolumn{1}{c}{Date}&\multicolumn{1}{c}{Country}&\multicolumn{1}{c}{Begin Date} &\multicolumn{1}{c}{Start Date}\\
  \hline
Argentina & Beunos Aires SE General Index & Dec-66 & Jan-17\\
Australia & Australia ASX All Ordinaries & Jan-58 & Jan-17\\
Bangladesh & Dhaka SE Index & Jan-90 & Jan-17\\
Canada & Canada S\&P/TSX 300 Composite & Jan-76 & Jan-17\\
Chile & Santiago IGBC General Index & Jan-75 & Jan-17\\
Colombia & Colombia IGBC General Index & Jan-92 & Jan-17\\
Ecuador & Ecuador Bolsa de Valores de Guayaquil (BVG) & Jan-94 & Jan-17\\
Egypt & Cairo SE Index & Dec-92 & Jan-17\\
Emerging Market & S\&P/IFC Emerging Markets Investable Composite & Jul-95 & Jan-17\\
Greece & Athens SE General Index & Oct-88 & Jan-17\\
India & Bombday SE SENSEX & Apr-79 & Jan-17\\
Indonesia & Jakarta SE Composite Index & Apr-83 & Jan-17\\
Iran & Tehran SE Price Index & Jan-95 & Jan-17\\
Israel & Tel Aviv 100 Index & May-87 & Jan-17\\
Japan & Tokyo SE Price Index (TOPIX) & Jan-53 & Jan-17\\
Latin America & Dow Jones Latin America Index & Jan-92 & Jan-17\\
Lithuania & Lithuania Lit-10 Index & Jan-99 & May-05\\
Malaysia & Malaysia KLSE Composite & Jan-80 & Jan-17\\
Nepal & Nepal NEPSE Stock Index & Jan-01 & Jan-17\\
Netherlands & Netherlands All-Share Price Index & Jan-80 & Jan-17\\
Pakistan & Karachi SE 100 Index & Jan-89 & Jan-17\\
Paraguay & Asuncion SE PDV General Index & Oct-93 & Sep-08\\
Peru & Lima SE General Index & Jan-82 & Jan-17\\
Philippines & Manila SE Composite Index & Jan-86 & Jan-17\\
Portugal & Oporto PSI-20 Index & Jan-86 & Jan-17\\
Singapore & Singapore FTSE ST Index & Jul-65 & Jan-17\\
South Korea & Korea SE Stock Price Index  & Jan-62 & Jan-17\\
Southeast Asia & Dow Jones Southeast Asia Index & Jan-92 & Jan-17\\
Spain & Madrid SE General Index & Aug-71 & Jan-17\\
Sri Lanka & Colombo SE All-Share Index & Dec-84 & Jan-17\\
Sweden & Sweden OMX Affarsvarlden General Index & Jan-80 & Jan-17\\
Taiwan & Taiwan SE Capitalization Weighted Index & Jan-67 & Jan-17\\
Thailand & Thailand SET General Index & Apr-75 & Jan-17\\
Tunisia & Tunisia SE Index & Dec-97 & Jan-17\\
Turkey & Instanbul IMKB 100 Price Index & Oct-87 & Jan-17\\
Ukraine & Ukraine PFTS OTC Index & Jan-98 & Jan-17\\
United Kingdom & UK FTSE All-Share Index & Nov-62 & Jan-17\\
United States & Dow Jones Industrial Average & Feb-1885 & Jan-17\\
Uruguay & Bolsa de Valores de Montevideo Index & Jan-08 & Jul-16\\
Venezuela & Dow Jones Venezuela Stock Index & Jan-92 & Jul-07\\
Venezuela & Caracas SE General Index & Jan-94 & Jan-17\\
World & MSCI World Price Index & Jan-76 & Jan-17\\
Zambia & Lusaka SE Index & Jan-02 & Apr-06\\
Zambia & Lusaka SE Index & Jul-11 & Jan-17\\
   \hline
   \hline
\end{tabular*}
\end{threeparttable}
\end{center}
\end{table}

To gain perspective on the relationship between irregular regime changes and the stock market, \autoref{fig:AV-DR} plots the absolute value of daily stock returns averaged across all events. Returns are for 250 trading days---approximately one calendar year---before and after regime changes.

The absolute value of returns on the event day (trading day 0) are significantly larger than on any other day. In addition, the magnitude of returns begins increasing just before the event day and remains high for a short period after. This suggests that financial volatility increases during the days surrounding regime changes, although it does not provide any evidence on mean returns.

\autoref{sec: Impact of Political Instability on Stock Returns} tests these results more formally. It will first analyze coups, assassinations and resignations separately and determine whether they increase mean returns. It will then combine all irregular regime changes and estimate the effect of regime changes on the variance of returns.

\begin{figure}[!ht]
\includegraphics{../figs/daily_mean_absreturn.pdf}
\caption{Absolute value of daily returns}
\label{fig:AV-DR}
\end{figure}


\section{Impact of Political Instability on Stock Returns} \label{sec: Impact of Political Instability on Stock Returns}
\subsection{Abnormal Returns} \label{subsec:abnormal-returns}
We begin our analysis by studying the effect of irregular regime changes on stock returns. We follow the standard event study methodology as presented by, among others, \citet{mackinlay1997event} and \citet{campbell1997econometrics}. Normal performance is measured with a constant mean return model\footnote{A constant mean return model is used instead of a market model in order to maximize the number of observations (plausible market indices such as the MSCI World Index and the S\&P/IFC Emerging Markets Investable Composite Index only begin in 1976 and 1995 respectively). That said, results are insensitive to the use of a market model.},
\begin{align} \label{eqn:market-model}
R_{it}=\mu_{i}+\epsilon_{it},
\end{align}
where $R_{it}$ is the logged return of national stock index $i$ on trading day $t$ and $\epsilon_{it}$ is the error term. We calculate abnormal returns (ARs), in an ``event window'' surrounding the date of each coup, $AR_{i\tau}=R_{i\tau}-\widehat{\mu}_i$, where $\tau$ is a date in the event window, and $\widehat{\mu}_i$ is estimated in an ``estimation window'' preceding the event window with \autoref{eqn:market-model}. We use a 41 day event window (i.e. 20 pre-event trading days, the event day, and 20 post-event trading days). The estimation window is the 250 trading days prior to the start of the event window. The abnormal returns are then used to generate cumulative abnormal returns (CARs) between event day $\tau_1$ and event day $\tau_2$: $CAR(\tau_1,\tau_2)=\sum_{\tau=\tau_1}^{\tau_2}AR_{i\tau}$.

We define the event date as the first trading day in which the market could have reacted to news of the event. For example, during the October 12, 1999 coup d'etat in Pakistan led by General Pervez Musharraf, the army announced that Prime Minister Nawaz Sharif had been dismissed after market hours at 10:15 pm. We code October 14th, the day in which the market re-opened, as the event day. When events occurred on weekends, we change the event date to the following Monday.

$(0,\tau-1)$ is used to denote the $\tau$-day period beginning with the event day and $(-1,\tau)$ to denote the negative $\tau$-day period beginning with the day prior to the event day. In other words, for cumulative abnormal returns prior to the event date, we aggregate backwards starting at the day of the event. For example, $CAR(-1,-2$) is the sum of the abnormal returns on event date $-1$ and event day $-2$.

We report abnormal returns separately for coups, assassinations and resignations because they may have distinct effects on stock returns. Standard errors and p-values are calculated using asymptotic t-statistics as in \citet{mackinlay1997event}.\footnote{It is appropriate to use the standard normal distribution to calculate test statistics because the length of the estimation window is sufficiently long (250 trading days).} 

\subsubsection{Coups} \label{subsec: Coups}
\autoref{tab:AR-coups} shows abnormal returns for national stock indices both preceding and following coup d'etat. \autoref{tab:AR-coups} contains all coups presented in \autoref{tab:regime-changes} with the exception of the Argentinian coup of March 24, 1976 and the partial/failed Venezuelan coup attempt of 2002. The March 24, 1976 Argentinian coup is excluded from our analysis because the stock market remained closed from March 24 to April 5, 1976, or a period of twelve days.\footnote{Treating this twelve day period as a single day CARs results in a positive abnormal return of 58\%, a fluctuation that seems qualitatively unreasonable.}  We later examine the Venezuelan coup attempt of 2002 in detail, but do not include it in the sample in \autoref{tab:AR-coups} as it does not match our definition of a successful coup. 

\begin{table}[!ht]
\caption{Abnormal returns following coups} \label{tab:AR-coups}
\vspace{-5pt}
\footnotesize
\begin{center}
\begin{threeparttable}
\begin{tabular*}{\textwidth}{l@{\extracolsep{\fill}}ld{4}d{4}d{4}d{4}d{4}d{2}}
  \hline
  \hline
\multicolumn{2}{c}{} & \multicolumn{3}{c}{Post-Event CAR} & \multicolumn{2}{c}{Pre-Event CAR} & \multicolumn{1}{c}{\multirow{2}{*}{Days to}}\\
\cmidrule(r){3-5} \cmidrule(r){6-7}
\multicolumn{1}{c}{Country} & \multicolumn{1}{c}{Event Date} & \multicolumn{1}{c}{(0,0)} & \multicolumn{1}{c}{(0,6)} & \multicolumn{1}{c}{(0,19)} & \multicolumn{1}{c}{(-1,-7)} & \multicolumn{1}{c}{(-1,-20)} & \multicolumn{1}{c}{rebound}\\
  \hline
\ExpandableInput{../tables/artable-coups-car.txt}
  \hline
\ExpandableInput{../tables/artable-coups-car-mean.txt}
   \hline
   \hline
\end{tabular*}
\scriptsize
Notes: Standard errors are in parentheses. ``Days to rebound'' is the number of trading days following a negative stock return for the national stock index to return to pre-event level (it is calculated if the price decreases on the event day, not if the event day abnormal return is negative). Returns are inflation adjusted. 
\end{threeparttable}
\end{center}
\end{table}

The average coup has a $-2.1\%$ event day AR. Event day ARs for the 1970 coup in Argentina, the 1991 coup in Thailand, the 1992 coup in Peru, and the 1999 coup in Pakistan are all negative and statistically different than zero. Moreover, all of these cases except Thailand have negative post-event CARs and pre-event CARs that are statistically indistinguishable from zero. In all of these cases, the coup in question either overthrew a democratically elected government or changed governance from one military ruler to another. The initial negative reaction followed by additional post-event negativity is consistent with the expected market reaction from a successful authoritarian coup followed by post-event consolidation of power. 

Event day ARs are negative for all coups except the 1971 coup in Argentina, the 2002 coup in Nepal, and the 2002 failed coup d'etat attempt against Hugo Chavez (\autoref{fig:AR-Ven}). These results provide evidence that coups can in fact result in positive abnormal returns. While the 1971 Argentinian coup did result in another military leader, it did so while calling for free and democratic elections and replaced a government that had adopted extreme protectionist economic policies. In fact, by 1973 Argentina had transitioned to a democracy.\footnote{Based on Center for System Peace Polity IV polity score of 6. Values of 6-10 are defined as democracies.} The 2002 coup in Nepal resulted in a monarchical restoration, but occurred after the country's prime minister postponed general elections. The ultimately failed Venezuelan coup against Hugo Chavez replaced a left-wing populist government with a new pro-business president.

\begin{figure}[!ht]
\includegraphics{../figs/venezuela_coup_attempt_2002.pdf}
\caption{Abnormal returns surrounding the 2002 Venezuelan coup d'etat attempt}
\label{fig:AR-Ven}
\end{figure}

The failed coup attempt against Hugo Chavez allows us to examine the effects of both pro-business and anti-business regime change separately from simple uncertainty because investors reacted to an expected regime change twice: first, when Chavez was ousted, and second, when he was reinstated. On the evening of April 11, 2002, coup plotters removed Chavez from office and later detained him. Pedro Carmona, a Venezuelan economist and business leader, was named the transitional President of Venezuela. Two days later, on April 13, 2002, a popular uprising led to Chavez's reinstatement as president. This provides an estimate of the market's valuation of a transition from the Chavez regime to the Carmona regime and its valuation of a transition from the Carmona regime back to the Chavez regime. By extension, it provides an estimate of the impact of a shift from a left-wing populist government to a pro business regime in an emerging market.

\autoref{fig:AR-Ven} provides graphical evidence on the effect of the coup attempt. The top panel shows CARs for the 10 days prior to and following the event, along with 95\% confidence intervals. The daily ARs and corresponding confidence intervals are displayed in the bottom panel. The abnormal return on April 12, the first trading day in which investors could react to the coup, was 10\%. The market reacted similarly, albeit in the opposite direction, to Chavez's reinstatement as president: the abnormal return on the next trading day, April 15, 2002, was -8\%.

The results in the figure are particularly striking given the discrepancy between the ARs on event days 0 and 1 and all other days. The only days on which the ARs are statistically different from zero is on event days 0 and 1 after the coup attempt. The almost 0\% 10-day CAR preceding the coup makes this an ideal case as it implies that investors were unaware of the coup plot. The unexpected nature of the event means that the abnormal returns capture the true value of the regime change from Chavez to Carmona more accurately than they otherwise would.

\subsubsection{Assassinations} \label{subsec: Assassinations}

The results in \autoref{tab:AR-ass} are produced from analyses identical to those table \autoref{tab:AR-coups} but for successful assassinations rather than coups. Like the majority of coups, there is evidence that assassinations decrease stock prices. The mean event day abnormal return is negative and statistically different than zero. However, the result is driven by five events: the shooting of U.S. President William McKinley on September 6, 1901; the assassination of U.S. President John F. Kennedy on November 22, 1963; the assassination of Indian Prime Minister Indira Gandhi on October 31, 1984; the suicide bombing that killed Sri Lankan president Ranasinghe Premadasa on May 1, 1993; and the assassination of Israeli Prime Minister Yitzhak Rabin on the evening of November 4, 1995.

\begin{table}[!ht]
\caption{Abnormal returns following assassinations} \label{tab:AR-ass}
\vspace{-5pt}
\footnotesize
\begin{center}
\begin{threeparttable}
\begin{tabular*}{\textwidth}{l@{\extracolsep{\fill}}ld{4}d{4}d{4}d{4}d{4}d{2}}
  \hline
  \hline
\multicolumn{2}{c}{} & \multicolumn{3}{c}{Post-Event CAR} & \multicolumn{2}{c}{Pre-Event CAR} & \multicolumn{1}{c}{\multirow{2}{*}{Days to}}\\
\cmidrule(r){3-5} \cmidrule(r){6-7}
\multicolumn{1}{c}{Country} & \multicolumn{1}{c}{Event Date} & \multicolumn{1}{c}{(0,0)} & \multicolumn{1}{c}{(0,6)} & \multicolumn{1}{c}{(0,19)} & \multicolumn{1}{c}{(-1,-7)} & \multicolumn{1}{c}{(-1,-20)} & \multicolumn{1}{c}{rebound}\\
  \hline
\ExpandableInput{../tables/artable-ass-car.txt}
  \hline
\ExpandableInput{../tables/artable-ass-car-mean.txt}
   \hline
   \hline
\end{tabular*}
\scriptsize
Notes: Standard errors are in parentheses. ``Days to rebound'' is the number of trading days following a negative stock return for the national stock index to return to pre-event level (it is calculated if the price decreases on the event day, not if the event day abnormal return is negative). Returns are inflation adjusted. 
\end{threeparttable}
\end{center}
\end{table}

These results are consistent with our hypothesis that the nature of the political event and its expected impact on policy matters. While the mean effect of assassinations is negative, it is smaller in magnitude than for coups. Unlike a coup, an assassination is typically an unexpected event that may not necessarily cause immediate change in economic policy. As such, we would expect CARs to be negative due to increased instability and uncertainty, but smaller in magnitude to a coup or resignation due to greater expectations of policy inertia.

There is no evidence of post or pre-event CARs in almost any of the assassinations. This is consistent with expectations as assassinations are typically not predictable. As with coups, the number of days that it took the stock market to rebound to pre-event levels is fairly low.\footnote{One exception is the assassination of William Mckinley in which the stock market didn't fully recover for 963 days, or almost 4 calendar years. However, this was likely caused by the Panic of 1901, which began when the stock market crashed on May 17th, 1901, and not by McKinley's death (although the assassination may have exacerbated the panic). In any case, the length of this time period is so long that we omitted it when calculating the mean days to rebound in the figure.}

\begin{table}[!ht]
\caption{Abnormal returns following resignations} \label{tab:AR-resignations}
\vspace{-5pt}
\footnotesize
\begin{center}
\begin{threeparttable}
\begin{tabular*}{\textwidth}{l@{\extracolsep{\fill}}ld{4}d{4}d{4}d{4}d{4}d{2}}
  \hline
  \hline
\multicolumn{2}{c}{} & \multicolumn{3}{c}{Post-Event CAR} & \multicolumn{2}{c}{Pre-Event CAR} & \multicolumn{1}{c}{\multirow{2}{*}{Days to}}\\
\cmidrule(r){3-5} \cmidrule(r){6-7}
\multicolumn{1}{c}{Country} & \multicolumn{1}{c}{Event Date} & \multicolumn{1}{c}{(0,0)} & \multicolumn{1}{c}{(0,6)} & \multicolumn{1}{c}{(0,19)} & \multicolumn{1}{c}{(-1,-7)} & \multicolumn{1}{c}{(-1,-20)} & \multicolumn{1}{c}{rebound}\\
  \hline
\ExpandableInput{../tables/artable-res-car.txt}
  \hline
\ExpandableInput{../tables/artable-res-car-mean.txt}
   \hline
   \hline
\end{tabular*}
\scriptsize
Notes: Standard errors are in parentheses. ``Days to rebound'' is the number of trading days following a negative stock return for the national stock index to return to pre-event level (it is calculated if the price decreases on the event day, not if the event day abnormal return is negative). Returns are inflation adjusted. 
\end{threeparttable}
\end{center}
\end{table}

\subsubsection{Resignations} \label{subsec: Resignations}

In contrast to coups and assassinations, abnormal returns following resignations are large and positive (see \autoref{tab:AR-resignations}). The mean event day abnormal return is over 4\% and the positive returns are persistent and grow larger over time (mean 20-day CAR $\approx$ 11\%). Furthermore, event day ARs are only negative and statistically significant at even the ten percent level in two out of the fifteen resignations (Pakistan on April 19, 1993 and Tunisia on June 31, 2011).

These results are consistent with the idea that the expected impact of a political event on economic policy is highly important to the directionality of returns. The positive event day abnormal return following resignations is not surprising since resignations typically occur because of poor performance and/or loss of authority. Leaders were often ousted following corruption charges, allegations of fraud, financial crises, and/or political violence. 

For example, consider Ferdinand Marcos' resignation from office as President of the Philippines in February 1986. Prior to his resignation, the Philippine regime was known for rampant corruption, crony capitalism, extreme inequality, high unemployment, failed import substitution industrialization policy, and oligarchic control of the economy \citep{overholt1986rise, traywick2014}. In fact, the Philippines was the least preferred site for foreign investment amongst East Asian capitalist economies and possessed one of the worst capital investment to economic output ratios in Asia \citep{overholt1986rise}. Marcos held a snap presidential election on February 7, 1986, in which he declared victory despite overwhelming evidence of electoral fraud. Public protests ensued, and two weeks later the military withdrew its support of the Marcos regime \citep{lee2009armed}. Marcos was replaced by his electoral opponent, Corazon Aquino, who had run on a platform of economic liberalization and elimination of crony capitalism \citep{villegas1987philippines}. 

\subsubsection{Public Protests} \label{subsec: Public Protests}
The resignations studied in this paper are those in which leaders left office because of poor performance, public discontent and popular protests. While the previous section showed that the resignations themselves had large effects on stock returns, it is not unreasonable to expect the political actions preceding the resignations to have similarly large effects on financial markets. Indeed, corporate investors in the 2013 MIGA \textit{World Investment and Political Risk} ranked civil disturbances as the fourth most concerning type of political risk.

A recent example of a popular uprising preceding a resignation is the 2011 Egyptian Revolution that resulted in the overthrow of President Hosni Mubarak's regime.\footnote{Abnormal returns for this event are not shown in \autoref{tab:AR-resignations} because the stock market was closed on the day of Mubarak's resignation.} Clashes between security forces and protestors led to the deaths of hundreds of citizens and injuries to thousands more. The uprising began on January 25, 2011 when millions of protestors demanded the overthrow of the Egyptian leadership. Examples of public discontent included demonstrations, marches, riots, non-violent civil disobedience, and labor strikes.

The short-term impact of the Egyptian Revolution on the economy was disastrous. As shown in \autoref{fig:CAR-Egypt}, abnormal returns on the Egyptian Stock Exchange Index (EGX 30) were around -7\% on January 26th and -10\% the day after. To prevent further decline during the uprising, the Egyptian Stock Exchanged closed at the end of trading on January 27th. President Mubarak resigned on February 11, but the market remained closed until March 23, when CARs declined by another 9\%, before rebounding slightly thereafter.

\begin{figure}[!ht]
\includegraphics{../figs/egypt-revolution-2011.pdf}
\caption{Cumulative abnormal returns during the Egyptian revolution}
\label{fig:CAR-Egypt}
\end{figure}

An important question is whether other popular uprisings have had similar adverse economic consequences. To examine this, we explore all resignations that were driven by significant public protests.\footnote{The set of resignations includes all those listed in either the Coup d'etat Events Handbook or the Archigos Version 4.1 data set with available financial data. In practice, this is the 2011 Egyptian Revolution and the list or resignations in \autoref{tab:AR-resignations}.} Public protests include popular demonstrations, riots, non-violent civil resistance and other forms of public discontent. These events are listed in \autoref{tab:protest-list} in the appendix.

The start and end dates in \autoref{tab:protest-list} are the dates that protests began and leader's resigned respectively. Resignations caused by popular uprisings were identified by examining the descriptions in the Coup d'etat Events Handbook and Archigos Version 4.1. Additional Lexis Nexis searches were used to verify these descriptions.

In \autoref{tab:protest-stocks}, we examine whether public protests influence stock prices. The variable \textit{Protest} is equal to 1 during the dates in which citizens participate in political activities demanding the resignation of the executive and 0 otherwise. Non-protest dates are the 250 days prior to the start dates and after the end dates listed in \autoref{tab:protest-list}.\footnote{The volatility estimates used as the dependent variable in column (4) are estimated on the 250 days prior to the start date, the protest dates, and the 250 days following the end date.}

Column (1) suggests that public protests have no effect on stock returns. However, this occurs because some political movements increase stock prices while others decrease them. As shown in column (2), the absolute value of stock returns are over 2\% higher during public protests. These estimates would be biased if protest dates are correlated with higher world or regional stock market indices. To address this potential confounder, column (3) controls for returns on the S\&P/IFC Emerging Markets Investable Composite Stock Index. The coefficient on \textit{Protest} barely changes and the absolute value of returns are still about 2\% higher during public protests. Finally, column (4) shows that stock volatility is more than 1 percentage point higher during political movements.\footnote{Volatility estimation methodology is described in detail in \autoref{subsec: Volatility}.}
 

We therefore find that both volatility and the absolute value of returns increase during times of protest. Similarly to coups, however, the direction of returns is dependent upon the nature of the protest in question. 

\begin{table}[!ht]
\caption{Effect of public protests on stock prices} \label{tab:protest-stocks}
\vspace{-5pt}
\footnotesize
\begin{center}
\begin{threeparttable}
\begin{tabular*}{\textwidth}{l@{\extracolsep{\fill}}d{1.3}d{1.3}d{1.3}d{1.3}}
  \hline
  \hline
\multicolumn{1}{c}{}&\multicolumn{1}{c}{Returns} & \multicolumn{2}{c}{Absolute Value of Returns}&\multicolumn{1}{c}{Volatility}\\
\cmidrule(r){2-2} \cmidrule(r){3-4} \cmidrule(r){5-5}
 & \multicolumn{1}{c}{(1)}&\multicolumn{1}{c}{(2)}&\multicolumn{1}{c}{(3)}&\multicolumn{1}{c}{(4)}\\
  \hline
\ExpandableInput{../tables/protest-regtable.txt}
   \hline
   \hline
\end{tabular*}
\scriptsize
Notes: Standard errors clustered by event are in parentheses.
\end{threeparttable}
\end{center}
\end{table}

\subsection{Volatility} \label{subsec: Volatility}
Although irregular regime changes have disparate effects on the direction of stock returns depending on expected policy outcomes, the results from \autoref{subsec:abnormal-returns} suggest that all of these political events increase financial volatility. However, since stock volatility is not directly observable, one must decide how to best estimate volatility. Our estimates are obtained from a generalized autoregressive conditional heteroskedasticity (GARCH) model estimated using 1000 pre-event days, the event day and 1000 post-event days. As in \citet{jensen2005market} and \citet{leblang2005government}, we use the GARCH (1,1) specification. In particular, for national stock index $i$,
\begin{align*}
R_{it}=\mu_i + \epsilon_{it},\hspace{1cm} \epsilon_{it}\sim \mathcal{N}\left(0,\sigma_{it}^2\right),
\end{align*}
where $\mu_i$ is a constant and,
\begin{align*}
\sigma_{it}^2&=\gamma_{i}+\alpha_{i}\epsilon_{i,t-1}^2+\beta_{i}\sigma_{i,t-1}^2.
\end{align*}
The key parameter of interest is the conditional variance, $\sigma_{it}^2$. The one-period-ahead volatility forecasts, $\sigma_{it}$, are larger when $\epsilon_{i,t-1}^2$ and $\sigma_{i,t-1}^2$ are larger. In other words, the model predicts that large shocks will be followed by other large shocks.

\autoref{fig:volatility} shows the mean volatility ($\overline{\sigma_t}$) estimates from the GARCH (1,1) model across all irregular regime changes for 250 trading days prior to and 250 days after each event. As expected, the volatility estimates stay between a narrow range at nearly all dates except those surrounding the regime change. Volatility appears to increase slowly just before the regime change, albeit not to a degree out of line with previous fluctuations in volatility. This may suggest that investors sometimes have information about the events before they occur. Nonetheless, there is still an enormous volatility jump on the day of the regime change. Volatility then decreases to normal levels within a month of the event.

\begin{figure}[htb]
\includegraphics{../figs/mean-volatility.pdf}
\caption{Mean of volatility estimates from GARCH(1,1) models}
\label{fig:volatility}
\end{figure}

\subsection{Robustness} \label{subsec: Robustness}
There are some potential concerns with the results in \autoref{subsec:abnormal-returns}. First, the abnormal returns could have been driven by factors unrelated to the regime changes.  Second, the true effects of regime changes on firm value may be underestimated if investors had apriori information. Third, the reported means are based on small sample sizes so confidence intervals based on normally distributed abnormal returns may be inappropriate.

We explore these concerns in two ways. First, we reestimate mean CARs on a set of time-shifted placebo dates, with means computed across all events for each type of regime change. We shift event dates surrounding the actual event date backwards and forwards in increments of five days (-20, 15, 10, 5, 0, 5, 10, 15, and 20 days). In addition, we extend the forward shifted event dates to one year (110, 195, 285, and 365 days) to capture dates that are likely to be completely unaffected by the regime change. 

\autoref{fig:mean-car-by-regime-change} compares mean CARs estimated using the actual event date (\autoref{subfig:mean-car-by-regime-change-actual}) to CARs estimated with the event date shifted 1-year (365 days) into the future (\autoref{subfig:mean-car-by-regime-change-placebo}). \autoref{subfig:mean-car-by-regime-change-actual}---which reproduces the tabular results presented in \autoref{subsec:abnormal-returns}---shows that assassinations and coups are associated with negative event day ARs while resignations are associated with positive event day ARs. In contrast, there are no discernible abnormal returns in \autoref{subfig:mean-car-by-regime-change-placebo}. The event day ARs are considerably smaller in magnitude and are not statistically different from zero for either coups or resignations. Moreover, while CARs estimated using the actual event date for resignations trend upwards following the event day, there is no consistent trend in the placebo analysis. \autoref{fig:mean-car-by-regime-change} therefore suggests that the main results are not merely an artifact of the data.  

\begin{figure}
\centering
\begin{subfigure}{\textwidth}
\centering
\includegraphics[max size={.8\textwidth}{.8\textheight}]{../figs/mean-car-by-regime-change-type.pdf}
\caption{Actual event date} \label{subfig:mean-car-by-regime-change-actual}
\end{subfigure}
\begin{subfigure}{\textwidth}
\centering
\includegraphics[max size={.8\textwidth}{.8\textheight}]{../figs/mean-car-by-regime-change-type-placebo.pdf}
\caption{Event date shifted forward by one year} \label{subfig:mean-car-by-regime-change-placebo}
\end{subfigure}
\caption{Mean cumulative abnormal returns by type of regime change}\label{fig:mean-car-by-regime-change}
\end{figure}

To ensure that results of the placebo test based on 1-year did not occur by chance, \autoref{fig:mean-event-day-ar-by-regime-change} plots event-day abnormal against the number of days shifted. There are a few instances in which there are statistically significant (at the 95\% level) ARs in the same direction as those on the actual event day, but they are always smaller in magnitude than the ARs estimated using the actual event date. Most of the statistically significant placebo estimates also occur when dates are shifted within the post-event window (days 0-20), a period during which stock returns remain volatile and, in the case of resignations, there is a consistent upward trend in the CARs. When dates are shifted forward further ($\geq$ 110 days), the event day AR is only statistically significant in one case (day 365 for assassinations). Overall, these results reinforce the main results: the ARs on the actual event date capture most of effect of the regime change, although effects can sometimes persist in the short event window following the event date. 

These figures also provide evidence on the extent to which regime changes appear to be unexpected. For instance, the CARs prior to assassinations and coups presented in \autoref{subfig:mean-car-by-regime-change-actual} tend to be close to zero, suggesting that investors were unaware that a negative event was likely to occur in the coming days. For resignations, CARs trend downward in the 10 days prior to the event; however, if investors were aware that a resignation were about to occur one would expect the pre-event CARs to be positive given the positive CARs observed in the post-event window. Overall, there is thus little evidence that the CARs in the post-event window are not capturing most of the effect caused by the regime changes.

Second, we create a synthetic control portfolio for each event based on the techniques introduced in \citet{abadie2003economic} and \citet{abadie2010synthetic}. Each country is given a weight which represents its influence in the synthetic control portfolio. The weight is chosen so that the daily returns and the variance of the daily returns of the control portfolio and the event country are most similar in the estimation window. The set of possible countries in the control portfolio consists of all countries listed in \autoref{tab:stock-list}.\footnote{See \autoref{Appendix} for a more formal explanation.}

Non-parametric statistical techniques that are free from distributional assumptions are used to address concerns about inferences from small sample sizes. We employ the sign and the rank tests, which are based on the sign and the rank of the event day ARs respectively.\footnote{See section 8 in \citet{mackinlay1997event} for more details.} Both tests are less influenced by departures from normality than statistics based on traditional t-tests such as those reported earlier in this paper. \autoref{tab:non-parametric} compares event day ARs as well as ``abnormal absolute returns'' between the event country and the synthetic control portfolio using the non-parametric methods discussed above. The ``abnormal absolute returns'' are abnormal returns for the absolute value of stock returns. This is done to combine events since resignations tend to increase returns while assassinations and coups tend to decrease them. The idea that the absolute value of returns might increase during irregular regime changes is similar to the finding that volatility increases and is consistent with \autoref{fig:AV-DR}.

\begin{table}[!ht]
\caption{Non-parametric tests of the impact of regime changes} \label{tab:non-parametric}
\vspace{-5pt}
\footnotesize
\begin{center}
\begin{threeparttable}
\begin{tabular*}{\textwidth}{l@{\extracolsep{\fill}}d{1.3}d{1.3}d{1.3}d{1.3}d{1.3}d{1.3}d{1.3}}
  \hline
  \hline
&\multicolumn{3}{c}{Regime Change Country}&\multicolumn{3}{c}{Synthetic Control Portfolio}&\multicolumn{1}{c}{\multirow{2}{*}{Wilcoxon}}\\
\cmidrule(r){2-4} \cmidrule(r){5-7}
&\multicolumn{1}{c}{Mean}&\multicolumn{1}{c}{Rank}&\multicolumn{1}{c}{Sign}
&\multicolumn{1}{c}{Mean}&\multicolumn{1}{c}{Rank}&\multicolumn{1}{c}{Sign}
&\multicolumn{1}{c}{Rank Test}\\
\multicolumn{1}{c}{Event Type}&\multicolumn{1}{c}{CAR (0,0)}&\multicolumn{1}{c}{p-value}&\multicolumn{1}{c}{p-value}
&\multicolumn{1}{c}{CAR (0,0)}&\multicolumn{1}{c}{p-value}&\multicolumn{1}{c}{p-value}
&\multicolumn{1}{c}{p-Value}\\
  \hline
\ExpandableInput{../tables/non-parametric-ar0.txt}
   \hline
   \hline
\end{tabular*}
\scriptsize
Notes: Estimates for assassinations do not include the assassination of U.S. president William McKinley in 1901 because no control portfolios are available. Two-sided 
\end{threeparttable}
\end{center}
\end{table}

As shown in \autoref{tab:non-parametric}, the mean event day abnormal returns for coups, assassinations and resignations are all statistically different from zero at the 1\% level using the rank test statistic and the abnormal returns for coups and assassinations are significant at at least the 10\% level using the sign test. In addition, abnormal absolute returns for all events are statistically significant at the 5\% level using both the rank and sign statistics. On the other hand, the event day abnormal returns for the control portfolio are never statistically different from zero at even the 10\% level. Finally, the difference in means between the regime change country and the control portfolio are statistically different from zero for coups (1\% level), assassinations (10\% level), resignations (5\% level), and all events combined (1\% level) when using two-sided p-values from the Wilcoxon rank test.\footnote{The Wilcoxon rank test is a non-parametric statistical technique that can be used to compare differences between matched samples.} In sum, the synthetic control and small sample tests therefore suggest that the main results are not a result of deviations from normality or confounding world events.

\section{Conclusion and Discussion}
Max Weber (\citeyear{weber1947theory}) developed the concept of ``charismatic leadership'' whereby a leader's authority stems from their charisma. Such rule is often found in autocracies with weak institutions and few constrains placed on the power of leaders. \citet{jones2005leaders} has shown that leaders can have particularly large impacts on economic growth in these settings. It is therefore not surprising that there is often a strong temptation to overthrow "bad" leaders, even by undemocratic means.

Coups and other types of regime changes remain common, highlighting their relevance for economic and political development. Our sample consists of 5 coups, 1 assassination, 7 resignations, and 7 instances of public protect since 2000. The Arab spring is perhaps the most notable, with protests in response to oppressive regimes and low living standards spreading throughout the Middle East in late 2010. There have also been a number of failed coups such as the 2002 coup attempt against Hugo Chavez and the 2016 failed coup in Turkey. Yet despite their frequency there is little evidence on their economic consequences.

This paper helps fill the evidence gap by using an event study approach that exploits daily returns of national stock market indices. This approach provides well-identified estimates of the economic effects of regime changes that are less susceptible to endogeneity bias than prior cross country studies. Furthermore, the large sample of political events in our study increases the generalizability of our findings.  

The results are consistent with the idea that perceptions of government competence and changes in government have large impacts on investor confidence. But although the effect of regime changes and protests on stock volatility is substantial, the effect on the direction of stock returns is less certain. Abnormal returns following resignations are large and positive (4\%), abnormal returns following assassinations are negative and smaller in magnitude (2\%), and abnormal returns following coups also tend to be negative (2\%). However, the returns surrounding the failed 2002 coup that temporarily ousted Hugo Chavez from power in Venezuela show that coups can sometimes have large positive economic effects, which is consistent with estimates from \citet{girardi2018institution} showing  the election of Allende in Chile and the subsequent coup replacing him increased stock market valuations. In total, the evidence consequently suggests that abnormal returns tend to be positive following resignations because the regime change likely replaced ``bad'' leaders, and coups can result in positive returns in settings where socialist governments are replaced with more pro-business ones. Conversely, autocratic coups may increase fear of capital appropriation or the implementation of poor economic policies, while assassinations may be merely regarded as highly uncertain events.  

As in any study, there are a number potential avenues for future research. First, more research is needed to identify the pro- and anti-market characteristics of regime changes. These may be the types of regime changes used in this paper or more nuanced characteristics. Ideally, researchers would be better able to predict the expected effects of future regime changes on stock returns. Additionally, it would be useful to identify the mechanisms through which regime changes effect stock returns. In the case of Chile,  \citet{girardi2018institution} shows that increasing stock market valuations were caused by changes in private property rights rather than economic growth prospects or wage costs. Research focusing on the mechanisms in other settings would help generalize these findings. Lastly, it would be helpful to determine the extent to which stock market returns translate to broader economic development outcomes. \citet{meyersson2016political} has made progress on this front by examining the impact of coups on a number of outcomes in addition to economic growth including investment, debt, inflation, infant mortality, and years of schooling. It would be fruitful to examine whether the direction of the effects of different types of regime changes on these outcomes are consistent with their stock market effects. Such research can help enlarge the body of evidence on the extent to which regime changes cause institutional and political change, and, in turn, have significant consequences for economic development.


\newpage

\appendix
\setcounter{table}{0}
\renewcommand\thetable{\Alph{section}.\arabic{table}}
\setcounter{figure}{0}
\renewcommand{\thefigure}{A\arabic{figure}}

\section{Appendix} \label{Appendix}
\begin{table}[!ht]
\caption{List of public protests preceding resignations} \label{tab:protest-list}
\vspace{-5pt}
\footnotesize
\begin{center}
\begin{threeparttable}
\begin{tabular*}{\textwidth}{l@{\extracolsep{\fill}}llll}
  \hline
    \hline
\multicolumn{1}{c}{Country}&\multicolumn{1}{c}{Name}&\multicolumn{1}{c}{Start Date}&\multicolumn{1}{c}{End Date}\\
  \hline
Philippines & EDSA 1/Yellow Revolution & 2/22/1986 & 2/25/1986\\
Bangladesh & Bangladeshi Spring of 1990 & 11/27/1990 & 12/7/1990\\
Thailand & Black May & 5/17/1992 & 5/20/1992\\
Indonesia & Indonesian Riots & 5/12/1998 & 5/21/1998\\
Philippines & EDSA II & 1/17/2001 & 1/20/2001\\
Argentina & Argentina Riots & 12/16/2001 & 12/20/2001\\
Ukraine & Orange Revolution & 11/22/2004 & 1/23/2005\\
Ecuador & Ecuadorian Protests & 4/13/2005 & 4/20/2005\\
Nepal & Nepalese People's Revolution & 4/6/2006 & 4/24/2006\\
Tunisia & Tunisian Revolution & 12/18/2010 & 1/14/2011\\
Egypt & Egyptian Revolution & 1/25/2011 & 2/11/2001\\
   \hline
   \hline
\end{tabular*}
\scriptsize
\end{threeparttable}
\end{center}
\end{table}

\begin{figure}[!ht]
\includegraphics{../figs/mean-ar-by-regime-change-type-placebo.pdf}
\caption{Time-shifted placebo sensitivity analysis of mean event day abnormal return by type of regime change}
\label{fig:mean-event-day-ar-by-regime-change}
\end{figure}

\subsection{Synthetic Control Portfolio}
Let $\boldsymbol{R}_{k}$ be the vector of returns for the event country in the estimation window, $\boldsymbol{R}_{-k}$ be the vector of returns for all other countries in the estimation window, $\boldsymbol{X}_1=(\boldsymbol{R}_{k},\var(\boldsymbol{R}_{k}))$, $\boldsymbol{X}_0=(\boldsymbol{R}_{-k},\var(\boldsymbol{R}_{-k}))$, and $\boldsymbol{W}_{-k}$ be a $((N-1) \times 1)$ vector of weights where $N$ is the number of countries listed in \autoref{tab:stock-list}. Then $\boldsymbol{W}^*$ is chosen to minimize $(\boldsymbol{X}_1-\boldsymbol{X}_0\boldsymbol{W})'\boldsymbol{V}(\boldsymbol{X}_1-\boldsymbol{X}_0\boldsymbol{W})$ subject to $w_i\geq0$ $(i = 1,2,\ldots,N-1)$ and $\sum_i^{N-1} w_i = 1$, and the vector $\boldsymbol{V}$ is chosen so that stock returns for the control portfolio during the estimation window are are close as possible to the event country.\footnote{See \citet{abadie2003economic} for further details.}

\newpage
\pdfbookmark[1]{References}{References}
\bibliography{bibliography}

\end{document} 